% ------------------------------------------------
\StartExtendedAbstract
% ------------------------------------------------

\ExtAbstractSummary{%
The summary is a short, informative abstract of no more than 250 words. References should not be cited. The summary should (1) state the scope and objectives of the research, (2) describe the methods used, (3) summarize the results, and (4) state the principal conclusions. Text of the summary should be 12 pt Times New Roman font, single-spaced and justified. A single line space should be left below the title `SUMMARY'. Leave a single line space above the key words listed below.
} % End of \ExtAbstractSummary{}

% ------------------------------------------------

\ExtAbstractChapter{INTRODUCTION}
The purpose of the introduction is to tell readers why they should want to read your thesis/ dissertation. This section should provide sufficient background information to allow readers to understand and evaluate the paper's results.

The introduction should (1) present the nature and scope of the problem, (2) review related literature, (3) describe the materials used and method(s) of the study, and (4) describe the main results of the study.

All text in the main body of the extended abstract should be 12 pt Times New Roman font, single-spaced and justified. Main headings are placed in the centre of the column, in capital letters using 12 pt Times New Roman Bold font. Subheadings are placed on the left margin of the column and are typed in 12 pt Times New Roman Bold font.

% ------------------------------------------------

\ExtAbstractChapter{MATERIALS AND METHODS}
There is flexibility as to the naming of the section (or sections) that provide information on the method(s) or theories employed. The methodology employed inthe work must be described in sufficient detail or with sufficient references so that the results could be duplicated.

Your materials should be organised carefully. Include all the data necessary to support your conclusions, but exclude redundant or unnecessary data.

% ------------------------------------------------

\ExtAbstractChapter{RESULTS AND DISCUSSION}
The results and discussion sections present your research findings and your analysis of those findings. The results of experiments can be presented as tables or figures.

% ------------------------------------------------

\ExtAbstractSection{Figures and Tables}
Figures may be integrated within the results section of the extended abstract, or they can be appended to the end of the written text. Figures should be black \& white. They should be no wider than the width of the A4 page.

Tables can be created within Word. As noted for figures above, if a table is to be placed within the text, it can be no wider than the width of the A4 page. Larger tables will need to be placed at the end of the abstract.

Figures and tables should be numbered according to the order they are referenced in the paper. Figures and tables should be referred to by their number in the text. When referring to figures and tables in the text, spell out and capitalize the word Figure or Table. All figures and tables must have captions.

% ------------------------------------------------

\ExtAbstractSection{Captions}
Captions should clearly explain the significance of the figure or table without reference to the text. Details in captions should not be restated in the text. Parameters in figure captions should be included and presented in words rather than symbols.

Captions should be placed directly above the relevant table and beneath the relevant figure. The caption should be typed in 12 pt Times New Roman Bold font. Spell out the word `Table' or `Figure' in full. An example table and a figure follow.

% ------------------------------------------------

\InsertTable
  [caption={Specifications of the engine}]
  {
    \begin{tabular}{llll}
    \hline
    Engine &  &  & OPEL Astra C16SE \\ \hline
    Displacement (cc) &  &  & 1598 \\
    Bore x stroke(mm x mm) &  &  & 79 x 81.5 \\
    Value mechanism &  &  & SOHC \\
    Number of valves &  &  & Intake 4, exhaust 4 \\
    Compression ratio &  &  & 9.8:1 \\
    Torque &  &  & 135/3400 Nm/rpm \\
    Power &  &  & 74/5800 kW/rpm \\
    Ignition sequence &  &  & 1-3-4-2 \\
    Spark plug &  &  & BPR6ES \\
    Fuel &  &  & 95 unleaded gasoline \\
    Cylinder arrangment &  &  & In-line 4 cylinders \\ \hline
    \end{tabular}
  } % End of  \InsertTable{}

\InsertFigure
  [scale=0.5,
    caption={HC emission as a function of equivalence ratio}]
  {tutorial--pre-PH-fork/abstract/pic/extended-abstract-2.jpg}

% ------------------------------------------------

\ExtAbstractChapter{CONCLUSION}
This section should include (1) the main points of your paper and why they are significant, (2) any exceptions to, problems with, or limitations to your argument, (3) agreements or disagreements with previously published work, (4) theoretical and practical implications of the work, and (5) conclusions drawn.

% ------------------------------------------------
\EndExtendedAbstract
% ------------------------------------------------
