%
% This file is part of the project of
% National Cheng Kung University (NCKU) Thesis/Dissertation Template in LaTex.
% This project is hold at
%     <https://github.com/wengan-li/ncku-thesis-template-latex>
% by Wen-Gan Li.
%
% This project is distributed in the hope of usefuling to someone,
% you can redistribute it and/or modify it under the terms of the
% Attribution-NonCommercial-ShareAlike 4.0 International.
%
% You should have received a copy of the
% Attribution-NonCommercial-ShareAlike 4.0 International
% along with this project.
% If not, see <http://creativecommons.org/licenses/by-nc-sa/4.0/legalcode.txt>.
%
% Please feel free to fork it, modify it, and try it.
% Have fun !!!
%

% ------------------------------------------------

\documentclass[12pt, a4paper, onecolumn, english]{report}

% ------------------------------------------------

% XeLaTex檢查點, 以要求必須使用XeLaTex來處理模版
\usepackage{ifxetex}
\ifxetex\else\errmessage{模版: 請使用XeLaTex來產生論文.}\stop\fi

% ------------------------------------------------

% 引用字體的基本設定
%
% No longer need \usepackage[T1]{fontenc} and
% \usepackage[utf8]{inputenc} when using XeLaTeX and LuaLaTeX as the engine.
%

% 引用fontspec以提供控制英文字型
\usepackage{fontspec}
\defaultfontfeatures{Ligatures=TeX} % To support LaTeX quoting style

% 引用xeCJK以提供控制中文字型
\usepackage{xeCJK}

% ------------------------------------------------

% 引用需要的LaTex packages

% Some base packages
\usepackage{geometry}
\usepackage{fp}
\usepackage{ifthen}
\usepackage{pgfkeys}
\usepackage{xparse}
\usepackage{amsmath}
\usepackage[framemethod=tikz]{mdframed}
\usepackage{url}
\usepackage{color}
\usepackage{etoolbox}

% For floats
% flafter package will make sure that the floats are
% not placed before their definition
\usepackage{flafter}

% For list
% Ref: <http://ftp.yzu.edu.tw/CTAN/macros/latex/contrib/enumitem/enumitem.pdf>
%\usepackage{enumitem}
%\setlist{noitemsep, nosep}

% For paragraphs
\usepackage{parskip}

% For line spacing
% Ref: <https://en.wikibooks.org/wiki/LaTeX/Paragraph_Formatting>
\usepackage{setspace}

% For PDF
\usepackage{hyperref}
\usepackage{pdfpages}

% For figure
\usepackage{graphicx}
\usepackage{caption}
\usepackage{subcaption}

% For table
\usepackage{array}
\usepackage{multirow}
\usepackage{booktabs}
\usepackage{diagbox}

% For comment
\usepackage{comment}

% For 目錄
\usepackage[tocgraduated]{./template/tocstyle}
\usetocstyle{standard}
%\setcounter{tocdepth}{4} % 目錄會顯示subsubsection


% For pseudocode
\usepackage{algorithm}
\usepackage[noend]{algpseudocode}
\algnewcommand\algorithmicswitch{\textbf{switch}}
\algnewcommand\algorithmiccase{\textbf{case}}
\algnewcommand\algorithmicdefault{\textbf{default}}
\algnewcommand\algorithmicbreak{\textbf{break}}
\algdef{SE}[SWITCH]{Switch}{EndSwitch}[1]{\algorithmicswitch\ #1\ }{\algorithmicend\ \algorithmicswitch}%
\algdef{SE}[CASE]{Case}{EndCase}[1]{\algorithmiccase\ #1:}{\algorithmicend\ \algorithmiccase}%
\algdef{SE}[CASE]{Default}{EndDefault}[0]{\algorithmicdefault:}{\algorithmicend\ \algorithmiccase}%
\algtext*{EndSwitch}%
\algtext*{EndCase}%
\algtext*{EndDefault}%
\def\Break{\algorithmicbreak}

% For theorem
\usepackage{amsthm}
\usepackage{amssymb}


\input{./template/command/command}

% --------------------------

% 學校排版 Arrangement style
\input{template/style/NCKU.tex}



\ifdefined\optDOI
\usepackage{draftwatermark}
% --- DOI 碼 ---
% 由2018年下學期時, 論文需要插入DOI碼顯示在右下方. DOI碼會在國立成功大學論文上傳系統中, 在上傳時提供.
% 但是2021以後,好像規定又便了。學生不必自己插入DOI碼. PH20210823
\DraftwatermarkOptions{
text={\textbf{doi:10.6844/ncku.latex.template.2019.Z00}},
color={[gray]{0}},
fontsize={0.5cm},
scale={0.58},
angle={0},
hpos={18.0cm}, hanchor=c,
vpos={28.4cm}, vanchor=m
}%% <--- DOI number
\fi


% ----------------------------------------------------------------------------

% --- 行距 ---
% 同學可自行設定每行的距離, 這邊是以放大縮小方式來使用.
% 所以是輸入 0.1, 0.5, 1, 1.0, 1.5, 2.0, 2 等數字.
% 預設的行距: 1.2

%\SetLineStretch{1.2}

% ----------------------------------------------------------------------------

% --- 封面上語言和名字顯示方式 ---
%
% \DisplayCoverInChi:  封面以全中文顯示
% \DisplayCoverInEng:  封面以全英文顯示
% 只能選擇其中一個, 但只有最後設定的一方有效
% 預設使用\DisplayCoverInEng
%
% 另外預設在封面上只會顯示中文或英文名字而已.
% 不論你是使用\DisplayCoverInChi或\DisplayCoverInEng,
% 使用\DisplayCoverPeoplesBothNames以設定同時顯示中英文名字.

%% \DisplayCoverInChi
%% \DisplayCoverInEng
\DisplayCoverPeoplesBothNames


% ----------------------------------------------------------------------------

% --- Date 日期 ---

% 封面日期是統一使用學位考試合格(口試合格單)單為主要參考日期 (年、月(學位考試通過日期)).
% 例如105年7月口試,則封面日期為 中華民國105年7月 或 2016年7月.

% --- 口試的日期 ---
% 設定西元的年月日, 會自動計算出民國的年份, 和英文的月份轉換
% 次序: {年份}{月份}{日}
% \SetOralDate{2016}{12}{31}

\SetOralDate{2019}{12}{31}

%--------------------------------------------------

% --- 論文封面上的日期 ---

% 如是你是國立成功大學的學生, 則封面日期直接使用口試日期, 故不需再另設定.
% 但如果你不是國立成功大學的學生, 那本模版則不清楚 貴學校所定的規範是否要分開, 故先保留這功能.

% 設定西元的年月, 會自動計算出民國的年份, 和英文的月份轉換
% 次序: {年份}{月份}
% \SetCoverDate{2019}{12}

\SetCoverDate{2019}{12}

% ----------------------------------------------------------------------------

% --- 系所 Department or Institute ---
%
% 設定你的系所名字, e.g:
% \SetDeptMath 數學系
% \SetDeptCSIE 資訊工程學系

\SetDeptCSIE

% ----------------------------------------------------------------------------

% --- 指導老師 Advisor(s) ---
% 在封面上預算了最多3位的空間
% 中文名字固定以 博士  為結尾
% 英文名字固定以 Dr. 為開頭

% 有3種可使用, 用來設定3位老師的名字
% \SetAdvisorNameX{老師的名字}{Professor's name} % 同時設定中英文名字
% \SetAdvisorChiNameX{老師的名字}                % 只設定中文名字
% \SetAdvisorEngNameX{Professor's name}         % 只設定英文名字
% (NameX為NameA, NameB, NameC)

% 使用\SetAdvisorNameA是必須的, 而如果你的指導教授有2或3位,
% 那只要增加\SetAdvisorNameB和\SetAdvisorNameC則可

%% \SetAdvisorNameA{A}{A}
%% \SetAdvisorNameB{B}{B}
%% \SetAdvisorNameC{C}{C}

% ----------------------------------------------------------------------------

% --- 學位考試論文證明書 Defense Certificate ---
% 使用範例版本 或 使用檔案 只能選擇其中一方

% 使用範例版本
\DisplayOralTemplate

% --- 範例版本的語言 ---
% 選擇你需要的範例
% (Only work with \DisplayOralTemplate)
% \DisplayOralChiTemplate    % 顯示中文範例版本
% \DisplayOralEngTemplate    % 顯示英文範例版本

\DisplayOralChiTemplate    % 顯示中文範例版本
\DisplayOralEngTemplate    % 顯示英文範例版本

% --- 口試委員 Committee member(s) ---
% 口試委員數量 (至少2位, 最多9位, 預設為9位)
% (Only work with \DisplayOralTemplate)
% 博士學位考試委員會置委員五人至九人
% 碩士學位考試委員會置委員三人至五人
% 口試委員人數含指導教授
\SetCommitteeSize{9}

%--------------------------------------------------

% 使用學位考試論文證明書圖片檔案
% 把你的圖片放在'context/oral'下
% 之後設定中英文版所對應是哪一個檔案
% 就算已啟用\DisplayOralImage,
% 但沒有填寫圖檔檔名的話, 都不會顯示出來.
% (例子用的'example-oral-chi.pdf'和'example-oral-eng.pdf'已放在'context/oral'中)

%\DisplayOralImage                % 顯示圖檔
%\SetOralImageChi{example-oral-chi.pdf}   % 中文版檔案
%\SetOralImageEng{example-oral-eng.pdf}   % 英文版檔案

% ----------------------------------------------------------------------------

% --- 關鍵字 Keyword ---
% 最多9個關鍵字
% 為了方便同學自行設定
% 故所產出來的PDF檔案中的關鍵字和內文摘要的關鍵字
% 可獨立個別設定

% \SetKeywords是設定所產出來的PDF中的Keyword項目
% 可同時填寫中英文
% e.g
% \SetKeywords{Keyword A (關鍵字 A)}{Keyword B (關鍵字 B)}{Keyword C (關鍵字 C)}
% 或單純中文或英文
% \SetKeywords{Keyword A}{Keyword B}{Keyword C}
% \SetKeywords{關鍵字 A}{關鍵字 B}{關鍵字 C}

\SetKeywords{NCKU Thesis/Dissertation template}{Graduate}{LaTex/XeLaTex}

% 摘要中的關鍵字
% 為了方便同學們能達到以下情況:
% a. 只寫中文版摘要
% b. 只寫英文版摘要
% c. 同時寫中英文版摘要
% 故中英文版的關鍵字都是可個別設定
% \SetAbstractChiKeywords: 用來設定中文版摘要的關鍵字
% \SetAbstractEngKeywords: 用來設定英文版摘要的關鍵字
% \SetAbstractExtKeywords: 用來設定英文延伸摘要的關鍵字 (只有你要編寫英文延伸摘要才需要設定)
% 所以只要使用你需要寫的版本則可.
% 但如果2個版本都要寫, 則2個都同時使用則可.
% 沒有填寫的話, 則摘要中的關鍵字部份是不會顯示出來.
%
% e.g
% \SetAbstractChiKeywords{關鍵字 A}{關鍵字 B}{關鍵字 C}
% \SetAbstractEngKeywords{Keyword A}{Keyword B}{Keyword C}
% \SetAbstractExtKeywords{Keyword A}{Keyword B}{Keyword C}
% 英文延伸摘要的關鍵字理應會跟英文版摘要的關鍵字是一樣,
% 但為了同學能編寫不同內容和關鍵字, 故可獨立設定.

\SetAbstractChiKeywords{國立成功大學畢業論文模版}{碩博士}{LaTex/XeLaTex}
\SetAbstractEngKeywords{NCKU Thesis/Dissertation Template}{Graduate}{LaTex/XeLaTex}
\SetAbstractExtKeywords{NCKU Thesis/Dissertation Template}{Graduate}{LaTex/XeLaTex}


% --- 目錄 Index ---
% 設定可獨立使用, 但只有最後設定的一方有效

% 標題文字語言 Language
% 目錄的標題文字使用預設的中文或是英文
% \IndexChiMode:  標題文字為中文
% \IndexEngMode:  標題文字為英文
% 預設的目錄標題為: 目錄 (中文) / Table of Contents (英文)
% 預設的表格目錄標題為: 表格 (中文) / List of Tables (英文)
% 預設的圖片目錄標題為: 圖片 (中文) / List of Figures (英文)
% 預設使用\IndexEngMode

%\IndexChiMode
\IndexEngMode

% Section (節)
\SetNumberingFormat[Section]{%
  BeginText = {}, EndText = {},
  TextAlign = {Left},
  CNumStyle = {}, SNumStyle = {},
  SepAtIndex = {}, SepBetweenCnS = {},
} % End of \SetNumberingFormat{}

% SubSection (小節)
\SetNumberingFormat[SubSection]{%
  BeginText = {}, EndText = {},
  TextAlign = {Left},
  CNumStyle = {}, SNumStyle = {}, SSNumStyle = {},
  SepAtIndex = {.}, SepBetweenCnS = {}, SepBetweenSnSS = {},
} % End of \SetNumberingFormat{}
%%END content that was in input from conf/conf.tex


% --------------------------
% 在 pdf 簡介欄裡填入相關資料
\ifdefined\optLinks
  \ifx \ThesisTitleChi \undefined
    \hypersetup
    {
      pdftitle  = {\ThesisTitleEng},
      pdfauthor = {\AuthorNameEng},
    }
  \else
    \hypersetup
    {
      pdftitle  = {\ThesisTitleEng\ (\ThesisTitleChi)},
      pdfauthor = {\AuthorNameEng\ (\AuthorNameChi)},
    }
  \fi

  \hypersetup
  {
    unicode     = true,
    pdfcreator  = {\UniversityNameEng},
    pdfproducer = {xelatex},
    pdfsubject  = {Academic Thesis},
  }

  \ifthenelse{\equal{\GetPDFKeywords}{\empty}}{}{%
    \hypersetup{pdfkeywords = {\GetPDFKeywords}}}
\fi


% ------------------------------------------------

% 一些會受到conf.tex中設定而影響的package或排版的設定

% Makes all pages the height of the text on that page.
% No extra vertical space is added.
%\raggedbottom

% Setup all custom numbering format
\SetupNumberingFormat

% 當所有的package都include完後, 才真正設定我們要的字型,
% 以清掉所有由package影響到的設定.
\InitDefaultFontType

%\setlength{\parindent}{4em}
%\usepackage{indentfirst}

\InitTheoremFormats
