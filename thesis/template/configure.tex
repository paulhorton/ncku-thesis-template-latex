%
% This file is part of the project of
% National Cheng Kung University (NCKU) Thesis/Dissertation Template in LaTex.
% This project is hold at
%     <https://github.com/wengan-li/ncku-thesis-template-latex>
% by Wen-Gan Li.
%
% This project is distributed in the hope of usefuling to someone,
% you can redistribute it and/or modify it under the terms of the
% Attribution-NonCommercial-ShareAlike 4.0 International.
%
% You should have received a copy of the
% Attribution-NonCommercial-ShareAlike 4.0 International
% along with this project.
% If not, see <http://creativecommons.org/licenses/by-nc-sa/4.0/legalcode.txt>.
%
% Please feel free to fork it, modify it, and try it.
% Have fun !!!
%

% ------------------------------------------------

\documentclass[12pt, a4paper, onecolumn, english]{report}

% ------------------------------------------------

% XeLaTex檢查點, 以要求必須使用XeLaTex來處理模版
\usepackage{ifxetex}
\ifxetex\else\errmessage{模版: 請使用XeLaTex來產生論文.}\stop\fi

% ------------------------------------------------

% 引用字體的基本設定
%
% No longer need \usepackage[T1]{fontenc} and
% \usepackage[utf8]{inputenc} when using XeLaTeX and LuaLaTeX as the engine.
%

% 引用fontspec以提供控制英文字型
\usepackage{fontspec}
\defaultfontfeatures{Ligatures=TeX} % To support LaTeX quoting style

% 引用xeCJK以提供控制中文字型
\usepackage{xeCJK}

% ------------------------------------------------

% 引用需要的LaTex packages

% Some base packages
\usepackage{geometry}
\usepackage{fp}
\usepackage{ifthen}
\usepackage{pgfkeys}
\usepackage{xparse}
\usepackage{amsmath}
\usepackage[framemethod=tikz]{mdframed}
\usepackage{url}
\usepackage{color}
\usepackage{etoolbox}

% For floats
% flafter package will make sure that the floats are
% not placed before their definition
\usepackage{flafter}

% For list
% Ref: <http://ftp.yzu.edu.tw/CTAN/macros/latex/contrib/enumitem/enumitem.pdf>
%\usepackage{enumitem}
%\setlist{noitemsep, nosep}

% For paragraphs
\usepackage{parskip}

% For line spacing
% Ref: <https://en.wikibooks.org/wiki/LaTeX/Paragraph_Formatting>
\usepackage{setspace}

% For PDF
\usepackage{hyperref}
\usepackage{pdfpages}

% For figure
\usepackage{graphicx}
\usepackage{caption}
\usepackage{subcaption}

% For table
\usepackage{array}
\usepackage{multirow}
\usepackage{booktabs}
\usepackage{diagbox}

% For comment
\usepackage{comment}

% For 目錄
\usepackage[tocgraduated]{./template/tocstyle}
\usetocstyle{standard}
%\setcounter{tocdepth}{4} % 目錄會顯示subsubsection


% For pseudocode
\usepackage{algorithm}
\usepackage[noend]{algpseudocode}
\algnewcommand\algorithmicswitch{\textbf{switch}}
\algnewcommand\algorithmiccase{\textbf{case}}
\algnewcommand\algorithmicdefault{\textbf{default}}
\algnewcommand\algorithmicbreak{\textbf{break}}
\algdef{SE}[SWITCH]{Switch}{EndSwitch}[1]{\algorithmicswitch\ #1\ }{\algorithmicend\ \algorithmicswitch}%
\algdef{SE}[CASE]{Case}{EndCase}[1]{\algorithmiccase\ #1:}{\algorithmicend\ \algorithmiccase}%
\algdef{SE}[CASE]{Default}{EndDefault}[0]{\algorithmicdefault:}{\algorithmicend\ \algorithmiccase}%
\algtext*{EndSwitch}%
\algtext*{EndCase}%
\algtext*{EndDefault}%
\def\Break{\algorithmicbreak}

% For theorem
\usepackage{amsthm}
\usepackage{amssymb}


%
% This file is part of the project of
% National Cheng Kung University (NCKU) Thesis/Dissertation Template in LaTex.
% This project is hold at
%     <https://github.com/wengan-li/ncku-thesis-template-latex>
% by Wen-Gan Li.
%
% This project is distributed in the hope of usefuling to someone,
% you can redistribute it and/or modify it under the terms of the
% Attribution-NonCommercial-ShareAlike 4.0 International.
%
% You should have received a copy of the
% Attribution-NonCommercial-ShareAlike 4.0 International
% along with this project.
% If not, see <http://creativecommons.org/licenses/by-nc-sa/4.0/legalcode.txt>.
%
% Please feel free to fork it, modify it, and try it.
% Have fun !!!
%

% ----------------------------------------------------------------------------
% 一些用來設定function和variable的command
% Some function and variable that let user use and configure
%
% 此處只是一些預設值和function
% 修改內容是在'conf/conf'
% ----------------------------------------------------------------------------

% Static variable and some provided API
\input{./template/command/cmd-common}
\input{./template/command/cmd-figure}
\input{./template/command/cmd-figures}
\input{./template/command/cmd-table}
\input{./template/command/cmd-oral}
\input{./template/command/cmd-equation}
\input{./template/command/cmd-ref}
\input{./template/command/cmd-keyword}
%
% This file is part of the project of
% National Cheng Kung University (NCKU) Thesis/Dissertation Template in LaTex.
% This project is hold at
%     <https://github.com/wengan-li/ncku-thesis-template-latex>
% by Wen-Gan Li.
%
% This project is distributed in the hope of usefuling to someone,
% you can redistribute it and/or modify it under the terms of the
% Attribution-NonCommercial-ShareAlike 4.0 International.
%
% You should have received a copy of the
% Attribution-NonCommercial-ShareAlike 4.0 International
% along with this project.
% If not, see <http://creativecommons.org/licenses/by-nc-sa/4.0/legalcode.txt>.
%
% Please feel free to fork it, modify it, and try it.
% Have fun !!!
%

% ----------------------------------------------------------------------------

% Some helper function about font

% Reference from
% <http://texdoc.net/texmf-dist/doc/latex/fontspec/fontspec.pdf>

% ----------------------------------------------------------------------------

\newcommand{\VarFontTypeTimesKaiu}{0}

% -------------------------------------------

% Type Noto Sans CJK (Eng & Chi)
\def \VarFontTypeNotoSansCJKEngFileNameNormal {NotoSansCJKtc-Medium.otf}
\def \VarFontTypeNotoSansCJKEngFileNameBold {NotoSansCJKtc-Bold.otf}
\def \VarFontTypeNotoSansCJKChiFileNameNormal {NotoSansCJKtc-Medium.otf}
\def \VarFontTypeNotoSansCJKChiFileNameBold {NotoSansCJKtc-Bold.otf}

\newcommand{\VarFontTypeNotoSansCJK}{1}

% -------------------------------------------

% Custom type
% Font files is needed to be provided
% Default it is the Type TimesKaiu
\def \VarFontTypeCustomEngFileNameNormal {times.ttf} % Default
\def \VarFontTypeCustomEngFileNameItalic {timesi.ttf} % Default
\def \VarFontTypeCustomEngFileNameBold {timesbd.ttf} % Default
\def \VarFontTypeCustomEngFileNameBoldItalic {timesbi.ttf} % Default
\def \VarFontTypeCustomChiFileNameNormal {kaiu.ttf} % Default
\def \VarFontTypeCustomChiFileNameItalic {} % Default
\def \VarFontTypeCustomChiFileNameBold {} % Default
\def \VarFontTypeCustomChiFileNameBoldItalic {} % Default



\pgfkeys
{
  /ParseCustomFontFiles/.is family, /ParseCustomFontFiles,
  default/.style =
  {
    NormalFont = \empty,
    ItalicFont = \empty,
    BoldFont = \empty,
    BoldItalicFont = \empty,
  },
  NormalFont/.estore in = \TmpValueNormalFont,
  ItalicFont/.estore in = \TmpValueItalicFont,
  BoldFont/.estore in = \TmpValueBoldFont,
  BoldItalicFont/.estore in = \TmpValueBoldItalicFont,
} % End of \pgfkeys{}

\newcommand{\SetCustomEngFontFiles}[1][\empty]
{%
  \renewcommand\VarFontUseType{\VarFontTypeCustom}
  %
  % Parse the input
  \pgfkeys{/ParseCustomFontFiles, default, #1}%
  %
  \ifthenelse{\equal{\TmpValueNormalFont}{\empty}}{}{%
    \renewcommand{\VarFontTypeCustomEngFileNameNormal}{%
        \TmpValueNormalFont}}%
  \ifthenelse{\equal{\TmpValueItalicFont}{\empty}}{}{%
    \renewcommand{\VarFontTypeCustomEngFileNameItalic}{%
        \TmpValueItalicFont}}%
  \ifthenelse{\equal{\TmpValueBoldFont}{\empty}}{}{%
    \renewcommand{\VarFontTypeCustomEngFileNameBold}{%
        \TmpValueBoldFont}}%
  \ifthenelse{\equal{\TmpValueBoldItalicFont}{\empty}}{}{%
    \renewcommand{\VarFontTypeCustomEngFileNameBoldItalic}{%
        \TmpValueBoldItalicFont}}%
} % End of \newcommand{}

\newcommand{\SetCustomChiFontFiles}[1][\empty]
{%
  \renewcommand\VarFontUseType{\VarFontTypeCustom}
  %
  % Parse the input
  \pgfkeys{/ParseCustomFontFiles, default, #1}%
  %
  \ifthenelse{\equal{\TmpValueNormalFont}{\empty}}{}{%
    \renewcommand{\VarFontTypeCustomChiFileNameNormal}{%
        \TmpValueNormalFont}}%
  \ifthenelse{\equal{\TmpValueItalicFont}{\empty}}{}{%
    \renewcommand{\VarFontTypeCustomChiFileNameItalic}{%
        \TmpValueItalicFont}}%
  \ifthenelse{\equal{\TmpValueBoldFont}{\empty}}{}{%
    \renewcommand{\VarFontTypeCustomChiFileNameBold}{%
        \TmpValueBoldFont}}%
  \ifthenelse{\equal{\TmpValueBoldItalicFont}{\empty}}{}{%
    \renewcommand{\VarFontTypeCustomChiFileNameBoldItalic}{%
        \TmpValueBoldItalicFont}}%
} % End of \newcommand{}

\newcommand{\VarFontTypeCustom}{10}

% -------------------------------------------

\pgfkeys
{
  /ParseFontOption/.is family, /ParseFontOption,
  default/.style =
  {
    NormalFont = \empty,
    ItalicFont = \empty,
    BoldFont = \empty,
    BoldItalicFont = \empty,
  },
  NormalFont/.estore in = \TmpValueNormalFont,
  ItalicFont/.estore in = \TmpValueItalicFont,
  BoldFont/.estore in = \TmpValueBoldFont,
  BoldItalicFont/.estore in = \TmpValueBoldItalicFont,
} % End of \pgfkeys{}


\newcommand*\FontDirPath{./template/fonts/} % Fix font path

\newcommand{\SetEngMainFont}[2][\empty]
{%
  % Parse the input
  \pgfkeys{/ParseFontOption, default, #1}%
  %
  \defaultfontfeatures[#2]{%
    Path = \FontDirPath,
    UprightFont = \TmpValueNormalFont
  }%
  %
  \ifthenelse{\equal{\TmpValueItalicFont}{\empty}}{}{%
    \defaultfontfeatures+[#2]{%
      ItalicFont = \TmpValueItalicFont}}%
  \ifthenelse{\equal{\TmpValueBoldFont}{\empty}}{}{%
    \defaultfontfeatures+[#2]{%
      BoldFont = \TmpValueBoldFont}}%
  \ifthenelse{\equal{\TmpValueBoldItalicFont}{\empty}}{}{%
    \defaultfontfeatures+[#2]{%
      BoldItalicFont = \TmpValueBoldItalicFont}}%
  %
  \setmainfont{#2}
} % End of \newcommand{}


\newcommand{\SetChiMainFont}[2][\empty]
{%
  % Parse the input
  \pgfkeys{/ParseFontOption, default, #1}%
  %
  \setCJKmainfont[%
    Path = \FontDirPath,
    UprightFont = \TmpValueNormalFont,
    AutoFakeBold = true,
    AutoFakeSlant = true,
  ]{#2}%
  %
  \ifthenelse{\equal{\TmpValueItalicFont}{\empty}}{}{%
    \addCJKfontfeatures*{ItalicFont = \TmpValueItalicFont}}%
  \ifthenelse{\equal{\TmpValueBoldFont}{\empty}}{}{%
    \addCJKfontfeatures*{BoldFont = \TmpValueBoldFont}}%
  \ifthenelse{\equal{\TmpValueBoldItalicFont}{\empty}}{}{%
    \addCJKfontfeatures*{BoldItalicFont = \TmpValueBoldItalicFont}}%
  %
  %\setCJKmathfont{#2}%
} % End of \newcommand{}


% -------------------------------------------

\newcommand*\UseFontStyleTimesKaiu{\fontspec{TimesKaiuEngFont}\CJKfontspec{TimesKaiuChiFont}}

\def \InitFontStyleTimesKaiu
{
  \SetEngMainFont[%
    NormalFont = times.ttf,%
    ItalicFont = timesi.ttf,%
    BoldFont = timesbd.ttf,%
    BoldItalicFont = timesbi.ttf,%
    ]{TimesKaiuEngFont}%
  \SetChiMainFont[%
    NormalFont = kaiu.ttf,%標楷體
    ]{TimesKaiuChiFont}%
}


% -------------------------------------------

\def \UseFontStyleNotoSansCJK
{
  \fontspec{NotoSansCJKEngFont}%
  \CJKfontspec{NotoSansCJKChiFont}%
} % End of \newcommand{}

\def \InitFontStyleNotoSansCJK
{
  \SetEngMainFont[%
    NormalFont = \VarFontTypeNotoSansCJKEngFileNameNormal,%
    BoldFont = \VarFontTypeNotoSansCJKEngFileNameBold,%
    ]{NotoSansCJKEngFont}%
  \SetChiMainFont[%
    NormalFont = \VarFontTypeNotoSansCJKEngFileNameNormal,%
    BoldFont = \VarFontTypeNotoSansCJKEngFileNameBold,%
    ]{NotoSansCJKChiFont}%
} % End of \newcommand{}

% -------------------------------------------

\def \UseFontStyleCustom
{
  \fontspec{CustomEngFont}%
  \CJKfontspec{CustomChiFont}%
} % End of \newcommand{}

\def \InitFontStyleCustom
{
  \SetEngMainFont[%
    NormalFont = \VarFontTypeCustomEngFileNameNormal,%
    ItalicFont = \VarFontTypeCustomEngFileNameItalic,%
    BoldFont = \VarFontTypeCustomEngFileNameBold,%
    BoldItalicFont = \VarFontTypeCustomEngFileNameBoldItalic,%
    ]{CustomEngFont}%
  \SetChiMainFont[%
    NormalFont = \VarFontTypeCustomChiFileNameNormal,%
    ItalicFont = \VarFontTypeCustomChiFileNameItalic,%
    BoldFont = \VarFontTypeCustomChiFileNameBold,%
    BoldItalicFont = \VarFontTypeCustomChiFileNameBoldItalic,%
    ]{CustomChiFont}%
} % End of \newcommand{}

% -------------------------------------------

\newcommand*{\FontTypeToUse}{\VarFontTypeTimesKaiu} % Default

\newcommand{\UseDefaultFontType}
{%
  \if \FontTypeToUse \VarFontTypeTimesKaiu%
    \UseFontStyleTimesKaiu%
  \fi%
%
  \if \FontTypeToUse \VarFontTypeNotoSansCJK%
    \UseFontStyleNotoSansCJK%
  \fi%

  \if \FontTypeToUse \VarFontTypeCustom%
    \UseFontStyleCustom%
  \fi%
} % End of \newcommand{}



\newcommand{\InitDefaultFontType}
{%
  \if \FontTypeToUse \VarFontTypeTimesKaiu%
    \InitFontStyleTimesKaiu%
  \fi%
%
  \if \FontTypeToUse \VarFontTypeNotoSansCJK%
    \InitFontStyleNotoSansCJK%
  \fi%

  \if \FontTypeToUse \VarFontTypeCustom%
    \InitFontStyleCustom%
  \fi%
} % End of \newcommand{}

\input{./template/command/cmd-numbering}
\input{./template/command/cmd-list}
\input{./template/command/cmd-spacing}

% Helper function for different page or chapter
\input{./template/command/cmd-thesis}
\input{./template/command/cmd-page}
%
% This file is part of the project of
% National Cheng Kung University (NCKU) Thesis/Dissertation Template in LaTex.
% This project is hold at
%     <https://github.com/wengan-li/ncku-thesis-template-latex>
% by Wen-Gan Li.
%
% This project is distributed in the hope of usefuling to someone,
% you can redistribute it and/or modify it under the terms of the
% Attribution-NonCommercial-ShareAlike 4.0 International.
%
% You should have received a copy of the
% Attribution-NonCommercial-ShareAlike 4.0 International
% along with this project.
% If not, see <http://creativecommons.org/licenses/by-nc-sa/4.0/legalcode.txt>.
%
% Please feel free to fork it, modify it, and try it.
% Have fun !!!
%

% Some helper function use in cover

% ----------------------------------------------------------------------------
\newcommand{\StartCover}
{
  %
  \singlespacing%
  %
  \StartNewPage
  %
  % 設定使用 無頁碼
  \thispagestyle{empty}
  %
  \EnableCoverPageStyle
  %
} % End of \newcommand{}

\newcommand{\EndCover}
{
  \DisableCoverPageStyle
  \EndOfPage
  \UseDefaultLineStretch
} % End of \newcommand{}
% ----------------------------------------------------------------------------

% ----------------------------------------------------------------------------

% --- Degree name 學位 ---
% thesis 是指論文的通稱
% dissertation 指的是博士的論文

% 碩士論文  Master's thesis
% 博士論文  Doctoral dissertation
\ifdefined\ForPhD%
\newcommand{\DegreeStringChi}{博士論文}
\newcommand{\DegreeStringEng}{Doctoral dissertation}
\else
\newcommand{\DegreeStringChi}{碩士論文}
\newcommand{\DegreeStringEng}{Master's thesis}
\fi

\begin{comment}
\newcommand{\ValueDegreeMaster}{0}
\newcommand{\ValueDegreePhd}{1}
\newcommand{\FlagDegreeType}{\ValueDegreePhd} % Default
\newcommand{\GetFlagDegreeType}{\FlagDegreeType}
\newcommand{\SetFlagDegreeType}[1]{\renewcommand{\FlagDegreeType}{#1}}

\newcommand{\VarDegreeChiName}{碩士/博士} % Default
\newcommand{\VarDegreeEngName}{Master / Doctor} % Default
\newcommand{\degreeThesisEname}{Master's Thesis / Doctoral Dissertation} % Default

\newcommand{\GetChiDegree}{\VarDegreeChiName}
\newcommand{\GetEngDegree}{\VarDegreeEngName}
\newcommand{\GetEngDegreeThesis}{\degreeThesisEname}
\newcommand{\SetChiDegree}[1]{\renewcommand{\VarDegreeChiName}{#1}}
\newcommand{\SetEngDegree}[1]{\renewcommand{\VarDegreeEngName}{#1}}
\newcommand{\SetEngDegreeThesis}[1]{\renewcommand{\degreeThesisEname}{#1}}

\newcommand{\PhdDegree}
{
  \SetFlagDegreeType{\ValueDegreePhd}
  \SetChiDegree{博士}
  \SetEngDegree{Doctor}
  \SetEngDegreeThesis{Doctoral Dissertation}
} % End of \newcommand{}

\newcommand{\MasterDegree}
{
  \SetFlagDegreeType{\ValueDegreeMaster}
  \SetChiDegree{碩士}
  \SetEngDegree{Master}
  \SetEngDegreeThesis{Master's Thesis}
} % End of \newcommand{}
\end{comment}

% ----------------------------------------------------------------------------

% --- Date 日期 ---

% \CoverDateNumInChi: 日期使用中文數字,
% 而不是阿拉伯數字,
% 故使用'\CoverDateNumInChi'可以顯示
% '第一章' 而不是 '中華民國 103 年 12 月 31 日'.
% \CoverDateNumInChi必須配合\DisplayCoverInChi來使用, 否則會無效.
%\CoverDateNumInChi

% --- 論文的日期 ---
\newcommand{\ThesisYear}{2014}  % Default
\newcommand{\ThesisMonth}{1}    % Default

\newcommand{\SetThesisDate}[2]{\SetThesisDate{#1}{#2}} % For backporting
\newcommand{\SetCoverDate}[2]
{
  \SetThesisTaiwanYear{#1}
  \renewcommand{\ThesisYear}{#1}
  \renewcommand{\ThesisMonth}{#2}
} % End of \newcommand{}

\newcommand{\GetThesisYear}{\ThesisYear}
\newcommand{\GetThesisYearInTaiwanYear}{\ThesisTaiwanYearResult}
\newcommand{\GetThesisMonth}{\ThesisMonth}
\newcommand{\GetThesisMonthNumInChi}{\zhnumber{\ThesisMonth}}
\newcommand{\GetThesisMonthInEng}{\GetMonthInEng{\ThesisMonth}}

% ---  口試的日期 ---
\newcommand{\OralChiYear}{101}      % Default
\newcommand{\OralChiMonth}{1}       % Default
\newcommand{\OralChiDay}{1}         % Default
\newcommand{\OralEngYear}{2014}     % Default
\newcommand{\OralEngMonth}{January} % Default
\newcommand{\OralEngDay}{1}         % Default

\newcommand{\GetOralChiYear}{\OralChiYear}
\newcommand{\GetOralYearInTaiwanYear}
{\SetThesisTaiwanYear{\OralEngYear}\ThesisTaiwanYearResult}
\newcommand{\GetOralYearInTaiwanYearNumInChi}
{\SetThesisTaiwanYear{\OralEngYear}\zhdigits{\ThesisTaiwanYearResult}}
\newcommand{\GetOralChiMonth}{\OralChiMonth}
\newcommand{\GetOralChiDay}{\OralChiDay}
\newcommand{\GetOralEngYear}{\OralEngYear}
\newcommand{\GetOralEngMonth}{\OralEngMonth}
\newcommand{\GetOralEngDay}{\OralEngDay}
\newcommand{\GetOralEngDayNumInChi}{\zhnumber{\OralEngDay}}

\newcommand{\SetOralChiDate}[3]
{
  \SetOralTaiwanYear{#1}
  \renewcommand{\OralChiYear}{\OralTaiwanYearResult}
  \renewcommand{\OralChiMonth}{#2}
  \renewcommand{\OralChiDay}{#3}
} % End of \newcommand{}

\newcommand{\SetOralEngDate}[3]
{
  \renewcommand{\OralEngYear}{#1}
  \renewcommand{\OralEngMonth}{\GetMonthInEng{#2}}
  \renewcommand{\OralEngDay}{#3}
} % End of \newcommand{}

\newcommand{\SetOralDate}[3]
{
  \SetOralChiDate{#1}{#2}{#3}
  \SetOralEngDate{#1}{#2}{#3}
} % End of \newcommand{}

% ----------------------------------------------------------------------------

% --- 學院 College, 系所 Department and Institute ---

% --------------------------- College ---------------------------
\newcommand{\VarCollegeChiName}{學院 C}
\newcommand{\VarCollegeEngName}{College of C}
\newcommand{\SetCollChiName}[1]{\renewcommand{\VarCollegeChiName}{#1}}
\newcommand{\SetCollEngName}[1]{\renewcommand{\VarCollegeEngName}{#1}}
\newcommand{\SetCollName}[2]
{
  \SetCollChiName{#1}
  \SetCollEngName{#2}
} % End of \newcommand{}

\newcommand{\GetCollChiName}{\VarCollegeChiName}
\newcommand{\GetCollEngName}{\VarCollegeEngName}

% --------------------------- Department ---------------------------
\newcommand{\VarDepartmentChiName}{A 系 / 所}
%\newcommand{\VarDepartmentEngName}{DeptA} % Short form of department
\newcommand{\VarDepartmentEngFullName}{Department / Insitute A} % Full name of department
\newcommand{\SetDeptChiName}[1]{\renewcommand{\VarDepartmentChiName}{#1}}
%\newcommand{\SetDeptEngShortName}[1]{\renewcommand{\VarDepartmentEngName}{#1}}
\newcommand{\SetDeptEngFullName}[1]{\renewcommand{\VarDepartmentEngFullName}{#1}}
\newcommand{\SetDeptName}[3]
{
  \SetDeptChiName{#1}
%  \SetDeptEngShortName{#2}
  \SetDeptEngFullName{#3}
} % End of \newcommand{}

\newcommand{\GetDeptChiName}{\VarDepartmentChiName}
\newcommand{\GetDeptEngName}{\VarDepartmentEngFullName}

% ----------------------------------------------------------------------------

% --- 指導老師 Advisor(s) ---
% 在封面上預算了最多3位的空間
% 中文名字固定以 博士 結尾
% 英文名字固定以 Dr. 開頭

%% \newcommand{\ AdvisorNameChiA}{X}
%% \newcommand{\ AdvisorNameEngA}{X}
%% \newcommand{\ AdvisorNameChiB}{}
%% \newcommand{\ AdvisorNameEngB}{}
%% \newcommand{\ AdvisorNameChiC}{}
%% \newcommand{\ AdvisorNameEngC}{}

%% \newcommand{\GetAdvisorChiNameA}{\ AdvisorNameChiA}
%% \newcommand{\GetAdvisorEngNameA}{\ AdvisorNameEngA}
%% \newcommand{\GetAdvisorChiNameB}{\ AdvisorNameChiB}
%% \newcommand{\GetAdvisorEngNameB}{\ AdvisorNameEngB}
%% \newcommand{\GetAdvisorChiNameC}{\ AdvisorNameChiC}
%% \newcommand{\GetAdvisorEngNameC}{\ AdvisorNameEngC}

%% \newcommand{\SetAdvisorChiNameA}[1]{\renewcommand{\ AdvisorNameChiA}{#1}}
%% \newcommand{\SetAdvisorEngNameA}[1]{\renewcommand{\ AdvisorNameEngA}{#1}}
%% \newcommand{\SetAdvisorChiNameB}[1]{\renewcommand{\ AdvisorNameChiB}{#1}}
%% \newcommand{\SetAdvisorEngNameB}[1]{\renewcommand{\ AdvisorNameEngB}{#1}}
%% \newcommand{\SetAdvisorChiNameC}[1]{\renewcommand{\ AdvisorNameChiC}{#1}}
%% \newcommand{\SetAdvisorEngNameC}[1]{\renewcommand{\ AdvisorNameEngC}{#1}}

%% \newcommand{\SetAdvisorNameA}[2]
%% {
%%   \SetAdvisorChiNameA{#1}
%%   \SetAdvisorEngNameA{#2}
%% } % End of \newcommand{}

%% \newcommand{\SetAdvisorNameB}[2]
%% {
%%   \SetAdvisorChiNameB{#1}
%%   \SetAdvisorEngNameB{#2}
%% } % End of \newcommand{}

%% \newcommand{\SetAdvisorNameC}[2]
%% {
%%   \SetAdvisorChiNameC{#1}
%%   \SetAdvisorEngNameC{#2}
%% } % End of \newcommand{}

% ----------------------------------------------------------------------------

% Use to create cover
\newcommand{\CreateCover}%
{
  \begin{document}
  \input{./template/cover/cover}
  \end{document}
} % End of \newcommand{}

% Use to include and display inner cover
\newcommand{\DisplayInnerCover}%
{
	\input{./template/cover/inner}
} % End of \newcommand{}

% ----------------------------------------------------------------------------

\newcommand{\ValueDisplayCoverLangEng}{0}
\newcommand{\ValueDisplayCoverLangChi}{1}
\newcommand{\VarDisplayCoverLang}{\ValueDisplayCoverLangEng}
\newcommand{\GetDisplayCoverLang}{\VarDisplayCoverLang}
\newcommand{\DisplayCoverInChi}{\renewcommand{\VarDisplayCoverLang}{\ValueDisplayCoverLangChi}}
\newcommand{\DisplayCoverInEng}{\renewcommand{\VarDisplayCoverLang}{\ValueDisplayCoverLangEng}}

% 日期顯示中文數字
\newcommand{\ValueDisplayCoverDateNumInNum}{0}
\newcommand{\ValueDisplayCoverDateNumInChi}{1}
\newcommand{\VarDisplayCoverDateNum}{\ValueDisplayCoverDateNumInNum}
\newcommand{\GetDisplayCoverDateNum}{\VarDisplayCoverDateNum}
\newcommand{\CoverDateNumInChi}{\renewcommand{\VarDisplayCoverDateNum}{\ValueDisplayCoverDateNumInChi}}

% ----------------------------------------------------------------------------

% Display Chinese and English name in english cover
\newcommand{\CoverDisplayNameChiEng}{0} % Default not display both
\newcommand{\SetCDBothName}{\renewcommand{\CoverDisplayNameChiEng}{1}}
\newcommand{\GetCDBothName}{\CoverDisplayNameChiEng}
\newcommand{\DisplayCoverPeoplesBothNames}{\SetCDBothName}

% A wrapper to handle \CDBothName{}
\newcommand{\CDBothName}{\DisplayCoverPeoplesBothNames}

% ----------------------------------------------------------------------------

% 顯示 '(初稿)' (中文版) 和 '(Draft)' (英文版) 在封面
\newcommand{\GetTextDraftChi}{(初稿)}
\newcommand{\GetTextDraftEng}{(Draft)}
\newcommand{\VarCoverDisplayDraft}{0} % Don't display in default
\newcommand{\EnableFlagDisplayDraft}{\renewcommand{\VarCoverDisplayDraft}{1}}
\newcommand{\DisplayDraft}{\EnableFlagDisplayDraft}
\newcommand{\GetFlagDisplayDraft}{\VarCoverDisplayDraft}

% ----------------------------------------------------------------------------

\input{./template/command/cmd-chapter}
\input{./template/command/cmd-abstract}
%
% This file is part of the project of
% National Cheng Kung University (NCKU) Thesis/Dissertation Template in LaTex.
% This project is hold at
%     <https://github.com/wengan-li/ncku-thesis-template-latex>
% by Wen-Gan Li.
%
% This project is distributed in the hope of usefuling to someone,
% you can redistribute it and/or modify it under the terms of the
% Attribution-NonCommercial-ShareAlike 4.0 International.
%
% You should have received a copy of the
% Attribution-NonCommercial-ShareAlike 4.0 International
% along with this project.
% If not, see <http://creativecommons.org/licenses/by-nc-sa/4.0/legalcode.txt>.
%
% Please feel free to fork it, modify it, and try it.
% Have fun !!!
%

% Some helper function use in extended abstract

% ----------------------------------------------------------------------------

\def\ValueEnableExtendedAbstractFigureTableControl{1}
\def\ValueDisableExtendedAbstractFigureTableControl{0}
\def\VarStartExtendedAbstractFigureTableControl{%
  \ValueDisableExtendedAbstractFigureTableControl}
\def\BeginExtendedAbstractFigureTableControl{%
  \renewcommand{\VarStartExtendedAbstractFigureTableControl}{%
    \ValueEnableExtendedAbstractFigureTableControl}}
\def\EndExtendedAbstractFigureTableControl{%
  \renewcommand{\VarStartExtendedAbstractFigureTableControl}{%
    \ValueDisableExtendedAbstractFigureTableControl}}
\def\GetStartExtendedAbstractFigureTableControl{%
  \VarStartExtendedAbstractFigureTableControl}

% ----------------------------------------------------------------------------

% Extended Abstract
\newcommand{\StartExtendedAbstract}
{%
  \singlespacing%
  %
  \StartNewPage%
  %
  % Add to "Table of Contents"
%  \addcontentsline{toc}{chapter}{Extended Abstract}
  \addcontentsline{toc}{chapter}{英文延伸摘要}%
  %
  % Set style of caption for figure and table
  \clearcaptionsetup{table}
  \clearcaptionsetup{figure}
  \UseTableCaptionExtendedAbstractStyle%
  \UseFigureCaptionExtendedAbstractStyle%
  %
  \BeginExtendedAbstractFigureTableControl%
  %
  \UseTableNameDefault%
  \UseFigureNameDefault%
  %
  % -----------------------------------------------------------------
  %
  \begin{minipage}[c][5cm][c]{\textwidth}
  \parbox{\textwidth}{\center\large\textbf{\ThesisTitleEng}}
  \vspace{0.3cm}%
  \center\normalsize\AuthorNameEng\par%
  \center\normalsize Dr. \GetAdvisorEngNameA\par%
  \ifthenelse{\equal{\GetAdvisorEngNameB}{\empty}}{}%
    {\center\normalsize Dr. \GetAdvisorEngNameB\par}
  \ifthenelse{\equal{\GetAdvisorEngNameC}{\empty}}{}%
    {\center\normalsize Dr. \GetAdvisorEngNameC\par}
  \center\normalsize\GetDeptEngName\par%
  \center\normalsize\GetCollEngName\par%
%  \center\normalsize\textit{\GetDeptEngName}\\%
%  \center\normalsize\textit{\GetCollEngName}\\%
  \end{minipage}%
  %
} % End of \newcommand{}

\newcommand{\EndExtendedAbstract}
{%
  \EndChapter%
  %
  \EndExtendedAbstractFigureTableControl%
  %
  \UseTableNameCustom%
  \UseFigureNameCustom%
  %
  % Reset figure and table counter to zero
  \setcounter{table}{0}%
  \setcounter{figure}{0}%
  %
  % Reset style of caption of figure and table
  \clearcaptionsetup{table}
  \clearcaptionsetup{figure}
  \UseTableCaptionDefaultStyle%
  \UseFigureCaptionDefaultStyle%
  \SetupNumberingFormat%
  %
  \UseDefaultLineStretch%
} % End of \newcommand{}

% Summary in Extended Abstract
\global\mdfdefinestyle{ExtAbstractSummaryStyle}{%
  linewidth=1pt, apptotikzsetting={%
    \tikzset{mdfbackground/.append style={opacity=0.4}}}%
} % End of \mdfdefinestyle{}
\newcommand{\ExtAbstractSummary}[1]
{%
  \par%
  \vspace{0.3cm}%
  \begin{mdframed}[style=ExtAbstractSummaryStyle]%
  \vspace{0.3cm}%
  \parbox{\textwidth}{\center\textbf{SUMMARY}}%
  \vspace{0.3cm}\par%
  #1%
  \EmptyLine%
%  \vspace{0.3cm}\par%
  \textbf{Keyword:} \GetAbstractExtKeywords%
  \end{mdframed}%
} % End of \newcommand{}

% Chapter in Extended Abstract
%\newcommand{\ExtAbstractChapter}[1]
\DeclareDocumentCommand{\ExtAbstractChapter}{+m +g} % Back-porting
{%
  \par%
  \vspace{0.5cm}%
  \centerline{\textbf{\MakeUppercase{#1}}}\par%
  \vspace{0.3cm}%
  \IfNoValueF{#2}{#2}%
} % End of \newcommand{}

% Section in Extended Abstract
%\newcommand{\ExtAbstractSection}[1]
\DeclareDocumentCommand{\ExtAbstractSection}{+m +g} % Back-porting
{%
  \par%
  \vspace{0.3cm}%
  \textbf{#1}\par%
  \vspace{0.1cm}%
  \IfNoValueF{#2}{#2}%
} % End of \newcommand{}

% ----------------------------------------------------------------------------

\input{./template/command/cmd-acknowledgments}
\input{./template/command/cmd-appendix}
\input{./template/command/cmd-bibliography}
\input{./template/command/cmd-index}
\input{./template/command/cmd-nomenclature}
\input{./template/command/cmd-theorem}

% Some function that use for school
\input{./template/command/cmd-college}
\input{./template/command/cmd-department}
%
% This file is part of the project of
% National Cheng Kung University (NCKU) Thesis/Dissertation Template in LaTex.
% This project is hold at
%     <https://github.com/wengan-li/ncku-thesis-template-latex>
% by Wen-Gan Li.
%
% This project is distributed in the hope of usefuling to someone,
% you can redistribute it and/or modify it under the terms of the
% Attribution-NonCommercial-ShareAlike 4.0 International.
%
% You should have received a copy of the
% Attribution-NonCommercial-ShareAlike 4.0 International
% along with this project.
% If not, see <http://creativecommons.org/licenses/by-nc-sa/4.0/legalcode.txt>.
%
% Please feel free to fork it, modify it, and try it.
% Have fun !!!
%

% Some helper function about watermark

% ----------------------------------------------------------------------------

% 學校圖案版的watermark
% 預設使用NCKU的浮水印
\newcommand{\VarWatermaskTextStyle}{%
  \vfill%
  \centering%
  \makebox(0,0){\rotatebox{45}{\textcolor[gray]{0.75}%
    {\fontsize{2.0cm}{2.0cm}\selectfont{\GetUniversityChiName}}}}%
  \vfill%
} % End of \newcommand{}

% 學校文字版的watermark
% 預設使用NCKU的浮水印
\newcommand{\VarWatermaskFigureStyle}{%
  \vfill%
  \centering%
  \includegraphics[]{./template/style/ncku/watermark-20160509_v2-a4.pdf}%
  \vfill%
} % End of \newcommand{}

% Prodive re-define APIs
\newcommand{\SetWatermaskTextStyle}[1]{%
  \renewcommand{\VarWatermaskTextStyle}{#1}%
} % End of \newcommand{}

\newcommand{\SetWatermaskFigureStyle}[1]{%
  \renewcommand{\VarWatermaskFigureStyle}{#1}%
} % End of \newcommand{}

\newcommand{\UseWatermarkFigureStyle}{%
  \AddToShipoutPicture{%
    \put(0,0){%
    \parbox[b][\paperheight]{\paperwidth}{%
      \VarWatermaskFigureStyle%
  }}}%
} % End of \newcommand{}

\newcommand{\UseWatermarkTextStyle}{%
  \AddToShipoutPicture{%
    \put(0,0){%
    \parbox[b][\paperheight]{\paperwidth}{%
      \VarWatermaskTextStyle%
  }}}%
} % End of \newcommand{}

\newcommand{\ClearWatermarkStyle}{\ClearShipoutPicture}

\newcommand{\ShowDOI}[1]{
  \SetWatermarkText{\textbf{#1}}
  \SetWatermarkColor{black}
  \SetWatermarkFontSize{0.5 cm}
  \SetWatermarkScale{0.58}
  \SetWatermarkAngle{0}
  \SetWatermarkLightness{0}
  \SetWatermarkHorCenter{18 cm}
  \SetWatermarkVerCenter{28.4 cm}
} % End of \newcommand{}

% 浮水印: Only add watermark when macro addWatermark is defined.  PH20210806
\ifdef{\addWatermark}{}{%
\renewcommand{\UseWatermarkTextStyle}{\relax}%
\renewcommand{\UseWatermarkFigureStyle}{\relax}%
}

% 設定預設使用學校浮水印 Watermark
\UseWatermarkFigureStyle

% ----------------------------------------------------------------------------


% ----------------------------------------------------------------------------


% --------------------------

% 學校排版 Arrangement style
%
% This file is part of the project of
% National Cheng Kung University (NCKU) Thesis/Dissertation Template in LaTex.
% This project is hold at
%     <https://github.com/wengan-li/ncku-thesis-template-latex>
% by Wen-Gan Li.
%
% This project is distributed in the hope of usefuling to someone,
% you can redistribute it and/or modify it under the terms of the
% Attribution-NonCommercial-ShareAlike 4.0 International.
%
% You should have received a copy of the
% Attribution-NonCommercial-ShareAlike 4.0 International
% along with this project.
% If not, see <http://creativecommons.org/licenses/by-nc-sa/4.0/legalcode.txt>.
%
% Please feel free to fork it, modify it, and try it.
% Have fun !!!
%
% ------------------------------------------------
%
% 台灣國立成功大學(NCKU)碩博士論文排版設定
%
% 基礎於陳朝鈞老師所提供的模版中的'utdiss.sty'進行了重新編寫和修改,
% 以對應模版設計和新版的LaTex (如LaTex 3).
% 有關修改內容可參考 <https://github.com/wengan-li/ncku-thesis-template-latex> 中
% 的ChangeLog.md.
% 對這檔案修改的版本和大約時間為:
%         v1.0.0 [Oct 14, 2014 ] 和更早的時間
%         v1.3.0 [Oct 26, 2016]
%         v1.4.1 [May 19, 2016]
%         v1.4.4 [May 25, 2016]
%         v1.4.5 [June 2, 2016]
%         v1.5.0 [Sep 11, 2016]
%         v1.5.2 [Jan 14, 2017]
%         v1.5.3 [Jul 11, 2018]
%         v1.5.7 [Jun 23, 2019]
%
% 原生自utdiss.sty
% utdiss.sty --- Version 2.1.1 (April 1992)
% Doctoral Dissertation Format Macros for The Univ. of Texas at Austin
%     By Young U. Ryu
%     Modified by Glenn G. Lai for LaTeX2e (May 1995)
%
% ------------------------------------------------

% 頁面邊界

% 國立成功大學各系(所)博碩士撰寫論文須知105.12.15 105學年度第2次教務會議修正
% 論文封面及內頁紙張規格:寬21 公分,長29.6 公分 (即A4尺寸) 80磅模造紙。
% 封面邊界:
%     直式:上23mm、下30mm、左20mm、右20mm
%     橫式:上37mm、下32mm、左28mm、右20mm
% 內頁邊界:
%     上23mm、下35mm(含頁碼)、左30mm、右25mm

% 預設邊界為: 內頁邊界
\geometry{a4paper,%
  paperwidth=21cm,paperheight=29.6cm,%
  top=2.3cm,bottom=3.5cm,left=3.0cm,right=2.5cm,%
  %showframe,%
  nohead%
} % End of \geometry{}

% API用來 '開始使用' 和 '停止使用' 封面邊界
\newcommand{\EnableCoverPageStyle}{%
  \newgeometry{%
    top=2.3cm,bottom=3cm,left=2cm,right=2cm,%
    nohead,nofoot%
  }%
} % End of \newcommand{}

\newcommand{\DisableCoverPageStyle}{\restoregeometry}

% ------------------------------------------------
% 預設前幾頁的頁碼是用羅馬數字
\pagenumbering{roman}

% ------------------------------------------------
% 設定學校圖案版的浮水印樣子
\SetWatermaskFigureStyle{%
  \vfill%
  \centering%
  \includegraphics[]{./template/style/ncku/watermark-20160509_v2-a4.pdf}%
  \vfill%
} % End of \SetWatermaskFigureStyle{}

% 設定學校文字版的浮水印樣子
\SetWatermaskTextStyle{%
  \vfill%
  \centering%
  \makebox(0,0){\rotatebox{45}{\textcolor[gray]{0.75}%
    {\fontsize{2.0cm}{2.0cm}\selectfont{\UniversityNameChi}}}}%
  \vfill%
} % End of \SetWatermaskTextStyle{}

% 設定預設使用學校浮水印的類型
\UseWatermarkFigureStyle
%\UseWatermarkTextStyle

% ------------------------------------------------
% 因每個學校所要求的內容不一樣, 故overwrite一些內容或APIs.

% 封面日期是統一使用學位考試合格(口試合格單)單為主要參考日期 (年、月(學位考試通過日期)).
\renewcommand{\SetOralDate}[3]
{%
  \SetOralChiDate{#1}{#2}{#3}%
  \SetOralEngDate{#1}{#2}{#3}%
  %
  \SetThesisTaiwanYear{#1}%
  \renewcommand{\ThesisYear}{#1}%
  \renewcommand{\ThesisMonth}{#2}%
} % End of \renewcommand{}

\renewcommand{\SetCoverDate}[3]
{%
%  \SetThesisTaiwanYear{\GetOralEngYear}
%  \renewcommand{\ThesisYear}{\GetOralEngYear}
%  \renewcommand{\ThesisMonth}{\GetOralEngMonth}
} % End of \renewcommand{}




\ifdefined\optDOI
\usepackage{draftwatermark}
% --- DOI 碼 ---
% 由2018年下學期時, 論文需要插入DOI碼顯示在右下方. DOI碼會在國立成功大學論文上傳系統中, 在上傳時提供.
% 但是2021以後,好像規定又便了。學生不必自己插入DOI碼. PH20210823
\DraftwatermarkOptions{
text={\textbf{doi:10.6844/ncku.latex.template.2019.Z00}},
color={[gray]{0}},
fontsize={0.5cm},
scale={0.58},
angle={0},
hpos={18.0cm}, hanchor=c,
vpos={28.4cm}, vanchor=m
}%% <--- DOI number
\fi


% ----------------------------------------------------------------------------

% --- 行距 ---
% 同學可自行設定每行的距離, 這邊是以放大縮小方式來使用.
% 所以是輸入 0.1, 0.5, 1, 1.0, 1.5, 2.0, 2 等數字.
% 預設的行距: 1.2

%\SetLineStretch{1.2}

% ----------------------------------------------------------------------------

% --- 封面上語言和名字顯示方式 ---
%
% \DisplayCoverInChi:  封面以全中文顯示
% \DisplayCoverInEng:  封面以全英文顯示
% 只能選擇其中一個, 但只有最後設定的一方有效
% 預設使用\DisplayCoverInEng
%
% 另外預設在封面上只會顯示中文或英文名字而已.
% 不論你是使用\DisplayCoverInChi或\DisplayCoverInEng,
% 使用\DisplayCoverPeoplesBothNames以設定同時顯示中英文名字.

%% \DisplayCoverInChi
%% \DisplayCoverInEng
\DisplayCoverPeoplesBothNames


% ----------------------------------------------------------------------------

% --- Date 日期 ---

% 封面日期是統一使用學位考試合格(口試合格單)單為主要參考日期 (年、月(學位考試通過日期)).
% 例如105年7月口試,則封面日期為 中華民國105年7月 或 2016年7月.

% --- 口試的日期 ---
% 設定西元的年月日, 會自動計算出民國的年份, 和英文的月份轉換
% 次序: {年份}{月份}{日}
% \SetOralDate{2016}{12}{31}

\SetOralDate{2019}{12}{31}

%--------------------------------------------------

% --- 論文封面上的日期 ---

% 如是你是國立成功大學的學生, 則封面日期直接使用口試日期, 故不需再另設定.
% 但如果你不是國立成功大學的學生, 那本模版則不清楚 貴學校所定的規範是否要分開, 故先保留這功能.

% 設定西元的年月, 會自動計算出民國的年份, 和英文的月份轉換
% 次序: {年份}{月份}
% \SetCoverDate{2019}{12}

\SetCoverDate{2019}{12}

% ----------------------------------------------------------------------------

% --- 系所 Department or Institute ---
%
% 設定你的系所名字, e.g:
% \SetDeptMath 數學系
% \SetDeptCSIE 資訊工程學系

\SetDeptCSIE

% ----------------------------------------------------------------------------

% --- 指導老師 Advisor(s) ---
% 在封面上預算了最多3位的空間
% 中文名字固定以 博士  為結尾
% 英文名字固定以 Dr. 為開頭

% 有3種可使用, 用來設定3位老師的名字
% \SetAdvisorNameX{老師的名字}{Professor's name} % 同時設定中英文名字
% \SetAdvisorChiNameX{老師的名字}                % 只設定中文名字
% \SetAdvisorEngNameX{Professor's name}         % 只設定英文名字
% (NameX為NameA, NameB, NameC)

% 使用\SetAdvisorNameA是必須的, 而如果你的指導教授有2或3位,
% 那只要增加\SetAdvisorNameB和\SetAdvisorNameC則可

%% \SetAdvisorNameA{A}{A}
%% \SetAdvisorNameB{B}{B}
%% \SetAdvisorNameC{C}{C}

% ----------------------------------------------------------------------------

% --- 學位考試論文證明書 Defense Certificate ---
% 使用範例版本 或 使用檔案 只能選擇其中一方

% 使用範例版本
\DisplayOralTemplate

% --- 範例版本的語言 ---
% 選擇你需要的範例
% (Only work with \DisplayOralTemplate)
% \DisplayOralChiTemplate    % 顯示中文範例版本
% \DisplayOralEngTemplate    % 顯示英文範例版本

\DisplayOralChiTemplate    % 顯示中文範例版本
\DisplayOralEngTemplate    % 顯示英文範例版本

% --- 口試委員 Committee member(s) ---
% 口試委員數量 (至少2位, 最多9位, 預設為9位)
% (Only work with \DisplayOralTemplate)
% 博士學位考試委員會置委員五人至九人
% 碩士學位考試委員會置委員三人至五人
% 口試委員人數含指導教授
\SetCommitteeSize{9}

%--------------------------------------------------

% 使用學位考試論文證明書圖片檔案
% 把你的圖片放在'context/oral'下
% 之後設定中英文版所對應是哪一個檔案
% 就算已啟用\DisplayOralImage,
% 但沒有填寫圖檔檔名的話, 都不會顯示出來.
% (例子用的'example-oral-chi.pdf'和'example-oral-eng.pdf'已放在'context/oral'中)

%\DisplayOralImage                % 顯示圖檔
%\SetOralImageChi{example-oral-chi.pdf}   % 中文版檔案
%\SetOralImageEng{example-oral-eng.pdf}   % 英文版檔案

% ----------------------------------------------------------------------------

% --- 關鍵字 Keyword ---
% 最多9個關鍵字
% 為了方便同學自行設定
% 故所產出來的PDF檔案中的關鍵字和內文摘要的關鍵字
% 可獨立個別設定

% \SetKeywords是設定所產出來的PDF中的Keyword項目
% 可同時填寫中英文
% e.g
% \SetKeywords{Keyword A (關鍵字 A)}{Keyword B (關鍵字 B)}{Keyword C (關鍵字 C)}
% 或單純中文或英文
% \SetKeywords{Keyword A}{Keyword B}{Keyword C}
% \SetKeywords{關鍵字 A}{關鍵字 B}{關鍵字 C}

\SetKeywords{NCKU Thesis/Dissertation template}{Graduate}{LaTex/XeLaTex}

% 摘要中的關鍵字
% 為了方便同學們能達到以下情況:
% a. 只寫中文版摘要
% b. 只寫英文版摘要
% c. 同時寫中英文版摘要
% 故中英文版的關鍵字都是可個別設定
% \SetAbstractChiKeywords: 用來設定中文版摘要的關鍵字
% \SetAbstractEngKeywords: 用來設定英文版摘要的關鍵字
% \SetAbstractExtKeywords: 用來設定英文延伸摘要的關鍵字 (只有你要編寫英文延伸摘要才需要設定)
% 所以只要使用你需要寫的版本則可.
% 但如果2個版本都要寫, 則2個都同時使用則可.
% 沒有填寫的話, 則摘要中的關鍵字部份是不會顯示出來.
%
% e.g
% \SetAbstractChiKeywords{關鍵字 A}{關鍵字 B}{關鍵字 C}
% \SetAbstractEngKeywords{Keyword A}{Keyword B}{Keyword C}
% \SetAbstractExtKeywords{Keyword A}{Keyword B}{Keyword C}
% 英文延伸摘要的關鍵字理應會跟英文版摘要的關鍵字是一樣,
% 但為了同學能編寫不同內容和關鍵字, 故可獨立設定.

\SetAbstractChiKeywords{國立成功大學畢業論文模版}{碩博士}{LaTex/XeLaTex}
\SetAbstractEngKeywords{NCKU Thesis/Dissertation Template}{Graduate}{LaTex/XeLaTex}
\SetAbstractExtKeywords{NCKU Thesis/Dissertation Template}{Graduate}{LaTex/XeLaTex}


% --- 目錄 Index ---
% 設定可獨立使用, 但只有最後設定的一方有效

% 標題文字語言 Language
% 目錄的標題文字使用預設的中文或是英文
% \IndexChiMode:  標題文字為中文
% \IndexEngMode:  標題文字為英文
% 預設的目錄標題為: 目錄 (中文) / Table of Contents (英文)
% 預設的表格目錄標題為: 表格 (中文) / List of Tables (英文)
% 預設的圖片目錄標題為: 圖片 (中文) / List of Figures (英文)
% 預設使用\IndexEngMode

%\IndexChiMode
\IndexEngMode

% Section (節)
\SetNumberingFormat[Section]{%
  BeginText = {}, EndText = {},
  TextAlign = {Left},
  CNumStyle = {}, SNumStyle = {},
  SepAtIndex = {}, SepBetweenCnS = {},
} % End of \SetNumberingFormat{}

% SubSection (小節)
\SetNumberingFormat[SubSection]{%
  BeginText = {}, EndText = {},
  TextAlign = {Left},
  CNumStyle = {}, SNumStyle = {}, SSNumStyle = {},
  SepAtIndex = {.}, SepBetweenCnS = {}, SepBetweenSnSS = {},
} % End of \SetNumberingFormat{}
%%END content that was in input from conf/conf.tex


% --------------------------
% 在 pdf 簡介欄裡填入相關資料
\ifdefined\optLinks
  \ifx \ThesisTitleChi \undefined
    \hypersetup
    {
      pdftitle  = {\ThesisTitleEng},
      pdfauthor = {\AuthorNameEng},
    }
  \else
    \hypersetup
    {
      pdftitle  = {\ThesisTitleEng\ (\ThesisTitleChi)},
      pdfauthor = {\AuthorNameEng\ (\AuthorNameChi)},
    }
  \fi

  \hypersetup
  {
    unicode     = true,
    pdfcreator  = {\UniversityNameEng},
    pdfproducer = {xelatex},
    pdfsubject  = {Academic Thesis},
  }

  \ifthenelse{\equal{\GetPDFKeywords}{\empty}}{}{%
    \hypersetup{pdfkeywords = {\GetPDFKeywords}}}
\fi


% ------------------------------------------------

% 一些會受到conf.tex中設定而影響的package或排版的設定

% Makes all pages the height of the text on that page.
% No extra vertical space is added.
%\raggedbottom

% Setup all custom numbering format
\SetupNumberingFormat

% 當所有的package都include完後, 才真正設定我們要的字型,
% 以清掉所有由package影響到的設定.
\InitDefaultFontType

%\setlength{\parindent}{4em}
%\usepackage{indentfirst}

\InitTheoremFormats
