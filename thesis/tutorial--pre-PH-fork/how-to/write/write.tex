% ------------------------------------------------
\StartChapter{LaTex編寫教學}
% ------------------------------------------------

% ------------------------------------------------
\StartSection{基本介紹 Introduction}{chapter:how-to:write:intro}
% ------------------------------------------------

這教學包含了原LaTex和本模版特有的語法的使用方式和例子. (真正完完整整的LaTex教學手冊可不只單單幾百頁的厚度, 所以減少大家的時間, 所以本模版教學只講一些幾乎大家100\%會需要使用的語法).

請注意原LaTex語法會以英文小寫來顯示(\verb|\aabbcc|); 而本模版特有的語法會以英文大小寫混合(\verb|\AaBbCc|, 第一個字必定以大寫來顯示), 由於這些特有語法\textbf{不是}原LaTex的語法, 所以不能直接應用在非本模版的LaTex檔案上.

抄襲就是學習的第一步 (如同我們小時候去抄襲父母走路一樣), 所以本模版有留下了一些範本 (在`./context'下)以方便大家開始第一步, 之後就要靠大家自己的努力和實作, 再加上自己的探索能力了.

%\newpage
有問題的話, 可以有以下的地方找尋答案 (請使用這順序):
\begin{enumerate}
  \item 請一步一步增加內容, 如發生錯誤, 就把剛剛新增的內容拿掉, 以找出錯誤的地方
  \item 直接研究在模版的LaTex寫法 (在 'example--pre-PH-fork/' 以下的所有檔案)
  \item 查問懂得LaTex的老師和同學
  \item 去LaTex的Wikibook \RefBib{web:latex:wikibooks}\\
        這邊有大量的例子, 但是這些例子都是獨立的, 所以潛在語法混合後的會發生沖突的可能性; 另外都十分推薦去讀 '大家來學LaTeX' \RefBib{web:latex:latex123}
  \item 請求Google老師
\end{enumerate}

另外, 如果覺得本教學還缺少了什麼說明, 請告知.

% ------------------------------------------------

% Section
\newpage% ------------------------------------------------
\StartSection{基本語法 Basic syntax}{chapter:how-to:write:basic}
% ------------------------------------------------

這邊會講解一些最基本的功能.

% ------------------------------------------------
% ------------------------------------------------
\StartSubSection{字體變化}

\begin{itemize}
  \item
  {
    正常

    這是文字 This is text
  } % End of \item{}

  \item
  {
    粗體

    寫法:
    \begin{DescriptionFrame}
    \verb|\textbf{這是文字 This is text}|
    \end{DescriptionFrame}

    效果: \textbf{這是文字 This is text}
  } % End of \item{}

  \item
  {
    斜体

    寫法:
    \begin{DescriptionFrame}
    \verb|\textit{這是文字 This is text}|
    \end{DescriptionFrame}

    效果: \textit{這是文字 This is text}\\
    (中文的斜体可能會並不太明顯)
  } % End of \item{}
\end{itemize}
% ------------------------------------------------


% ------------------------------------------------
\newpage% ------------------------------------------------
\StartSubSection{清單 List Structures}

  日常的清單主要有3種:

\begin{itemize}
  \item
  {
    數字

    可以有2種寫法, 使用\verb|\item xxxx|來只寫一行, 或是用\verb|{...}|可把內容包起來.\\

    \begin{DescriptionFrame}
    \begin{verbatim}
      \begin{enumerate}
      \item Item1

      \item Item2

      \item
      {
        Item3

        Item3's context
      }

      \item
      {
        Item4

        Item4's context
      }
      \end{enumerate}
    \end{verbatim}
    \end{DescriptionFrame}

    效果:
    \begin{enumerate}
      \item Item1

      \item Item2

      \item
      {
        Item3

        Item3's context
      }

      \item
      {
        Item4

        Item4's context
      }
    \end{enumerate}
  } % End of \item{}

  \newpage
  \item
  {
    符號

    \begin{DescriptionFrame}
    \begin{verbatim}
      \begin{itemize}
      \item Item1

      \item Item2

      \item
      {
        Item3

        Item3's context
      }

      \item
      {
        Item4

        Item4's context
      }
      \end{itemize}
    \end{verbatim}
    \end{DescriptionFrame}

    效果:
    \begin{itemize}
      \item Item1

      \item Item2

      \item
      {
        Item3

        Item3's context
      }

      \item
      {
        Item4

        Item4's context
      }
    \end{itemize}
  } % End of \item{}

  \newpage
  \item
  {
    文字

    可以有2種寫法, 使用\verb|\item[xxxx] xxxx|來只寫一行,\\
    或是用\verb|\hfill \\|把內容放到第2行才開始.\\

    \begin{DescriptionFrame}
    \begin{verbatim}
      \begin{description}
      \item[Item1] Item1's context
      \item[Item2] Item2's context
      \item[Item3] \hfill \\
        Item3's context
      \end{description}
    \end{verbatim}
    \end{DescriptionFrame}

    效果:
    \begin{description}
      \item[Item1] Item1's context
      \item[Item2] Item2's context
      \item[Item3] \hfill \\
      Item3's context
    \end{description}
  } % End of \item{}

  \newpage
  \item
  {
    巢狀表單

    表單應該最多只會用到第4層, 但是其實當你需要用到第3層時, 這時候你應該考慮的不是怎使用表單, 而是要怎換另外一種寫法了.\\

    \begin{DescriptionFrame}
    \begin{verbatim}
      \begin{enumerate}
        \item
        {
          Level-1 Item 1
          \begin{enumerate}
            \item Nested Item 1

            \item
            {
              Level-2 Item 2

              \begin{enumerate}
              \item
              {
                Level-3 Item 1
                \begin{enumerate}
                  \item Level-4 Item 1
                  \item Level-4 Item 2
                \end{enumerate}
              }
              \item Level-3 Item 2
              \end{enumerate}
            }
          \end{enumerate}
        }
      \end{enumerate}

      \begin{itemize}
        \item
        {
          Level-1 Item 1

          \begin{itemize}
            \item
            {
              Level-2 Item 2
              \begin{itemize}
                \item Level-3 Item 1
                \item Level-3 Item 2
              \end{itemize}
            }
            \item Level-2 Item 2
          \end{itemize}
        }
      \end{itemize}
    \end{verbatim}
    \end{DescriptionFrame}

    效果:
    \begin{enumerate}
      \item
      {
        Level-1 Item 1
        \begin{enumerate}
          \item Nested Item 1

          \item
          {
            Level-2 Item 2

            \begin{enumerate}
              \item
              {
                Level-3 Item 1

                \begin{enumerate}
                  \item Level-4 Item 1
                  \item Level-4 Item 2
                \end{enumerate}
              }

              \item Level-3 Item 2
            \end{enumerate}
          }
        \end{enumerate}
      }
    \end{enumerate}

    \begin{itemize}
      \item
      {
        Level-1 Item 1

        \begin{itemize}
        \item
        {
          Level-2 Item 2

          \begin{itemize}
          \item Level-3 Item 1
          \item Level-3 Item 2
          \end{itemize}
        }

        \item Level-2 Item 2
        \end{itemize}
      }
    \end{itemize}
  } % End of \item{}
\end{itemize}
% ------------------------------------------------


% ------------------------------------------------
\newpage% ------------------------------------------------
\StartSubSection{標記 Label}{chapter:how-to:write:label}
標記(Label)是指給某項東西(如圖, 表格, 段落, chapter等)一個用來記憶的名字, 主要用來在引用時可以用來指定它. 使用方式是:\\

  \begin{DescriptionFrame}
  \begin{verbatim}
    \label{ ... some text here for your label ...} % 設定Label

    e.g
    \label{fig:introduction:fig1} % 設定Label
    \RefTo{fig:introduction:fig1} % 引用Label
  \end{verbatim}
  \end{DescriptionFrame}

\noindent Label的名字是可以任何輸入的文字, 但是為了方便記憶, 會固定以一個名字起頭, 再以段落/章節的方式來分隔.\\

\noindent 在例子中`fig:introduction:fig1':\\
以`fig'起頭: 即是目標是一張圖像(figure).\\
以`introduction'為章節: 即是目標放在introduction這一章中.\\
最後`fig1': 這張圖像的名字為`fig1'.\\

\noindent 同樣其他方便記憶的目標起頭例如: `website', `table', `chapter', `section', `paper', etc.\\

\noindent 本模版提供的一些功能內, 已經把這功能包含進來了.

\newpage
\StartSubSection{引用 Reference}
因為原本LaTex的引用語法可以引用很多東西, 所以可能會混亂不知道自己在引用什麼, 故本模版提供幾個語法來取代那些語法. (但是如果你是懂得原LaTex的寫法(\verb|\ref{}, \cite{}, etc.|), 都可以直接使用原本的寫法, 其實是同一個東西.)\\

  \begin{DescriptionFrame}
  \begin{verbatim}
    引用 公式(Equation)
    \RefEquation{...}   直接顯示章節和它的號碼, 如: X.X
    \RefEquationB{...}  顯示時多了'()', 如: (X.X)

    引用 參考資料(References)
    \RefBib{...}   顯示號碼, 會加上'[]', 如: [X]

    引用 頁碼
    \RefPage{...}  顯示目標的頁碼, 如: X

    引用 其他任何的東西: 如圖片, 表格,
          chapter, section, subsection, etc.
    \RefTo{...}
      顯示章節和它的號碼, 如: X.X
      所以要手動在引用部份加上 fig, table, chap等一些字眼
  \end{verbatim}
  \end{DescriptionFrame}

由於label寫在LaTex中, 而產生出來的後的文件是看不到的, 所以沒法簡單講解來說明, 所以可以參考後面的一些章節, 其內容會有一些例子會方便理解.

例子:
\begin{itemize}
  \item 圖片 - 可參考P. \RefPage{fig:example:mi2:mfig}.

  \item 表格 - 可參考P. \RefPage{chapter:how-to:write:table:label-example}.

  \item 公式(Equation) - 可參考P. \RefPage{chapter:how-to:write:equation:label-example}.
\end{itemize}

% ------------------------------------------------


% ------------------------------------------------
\newpage% ------------------------------------------------
\StartSubSection{註解 Comment}{chapter:how-to:write:comment}
% ------------------------------------------------

編寫任何內容時, 都會有一些作輔助用的內容, 這些內容正常不一定是用來顯示給別人看, 而是給自己作一些記憶用的.\\

但是在Word中所寫的任何內容, 正常都是寫來公開的, 而一些個人後備輔助用的資料就會寫在另一個檔案中; 但在LaTex中可以一同把這些資料寫在同一個檔案中, 但可指定不顯示, 這些叫註解(Comment).\\

\begin{DescriptionFrame}
  \begin{verbatim}
    單行註解 (在第一個字使用'%'即可)

    % 註解內容 1
    % 註解內容 2
    顯示內容 1
       ...
    顯示內容 2
       ...
    多行註解 (把一個範圍內的內容為註解)

    \begin{comment}
    % 註解內容 1
    % 註解內容 2
    \end{comment}
    顯示內容 1
       ...
    顯示內容 2
       ...
  \end{verbatim}
\end{DescriptionFrame}


% ------------------------------------------------
\newpage% ------------------------------------------------
\StartSubSection{引用別的LaTex檔}

正常在編寫Word時, 都會把所有內容寫在同一個.doc中 (當然你都可能原本就喜好分開檔案來寫), 但在LaTex中這行為就不常見, 當內容很巨量的時候就更不用講, 這本模版更是其一例子.\\

  \begin{DescriptionFrame}
  \begin{verbatim}
    引用的方式
    \input{ ... 檔案位置 ... }

    如現在你的檔案為:
    thesis.tex (主檔案)
    a.tex
    b.tex

    那要引用a.tex和b.tex時
    在thesis.tex中要寫
    \input{./a.tex}
    \input{./b.tex}
  \end{verbatim}
  \end{DescriptionFrame}

如果還是不明白的話, 可以參考`tutorial--pre-PH-fork'中的引用方式.

% ------------------------------------------------


\newpage% ------------------------------------------------
\StartSection{章節 Chapter/Section}{chapter:how-to:write:chapter-section}
% ------------------------------------------------

編寫任何的文章, 都會使用不同的章節來把內容進行分區. 例如這模版預設的樣子為:
\begin{DescriptionFrame}
  \vspace{0.2cm}
  \centerline{\LARGE Chapter X}
  \vspace{0.3cm}
  \centerline{\LARGE 這是標題}

  \vspace{0.5cm}
  \mbox{\Large X.1 節標題}\\
  \mbox{\hspace{1.2cm}內容 ...}

  \vspace{0.3cm}
  \mbox{\large X.1.1 小節標題}\\
  \mbox{\hspace{1.2cm}內容 ...}

  \vspace{0.3cm}
  \mbox{\large 小小節標題}\\
  \mbox{\hspace{1.2cm}內容 ...}
\end{DescriptionFrame}

所以針對這些功能, 本模版提供:
\begin{DescriptionFrame}
  \begin{verbatim}
    主要章節
    Title: 標題 (必填)
    Label: 標簽 (選填)
    \StartChapter{ Title }{ Label }
    \EndChapter % 用來保證你的內容在這Chapter內

    節
    Title: 標題 (必填)
    Label: 標簽 (選填)
    \StartSection{ Title }{ Label }

    小節
    Title: 標題 (必填)
    Label: 標簽 (選填)
    \StartSubSection{ Title }{ Label }

    小小節
    Title: 標題 (必填)
    Label: 標簽 (選填)
    \StartSubSubSection{ Title }{ Label }
  \end{verbatim}
\end{DescriptionFrame}

所以針對剛剛的例子, 它的LaTex寫法為:\\

\begin{DescriptionFrame}
  \begin{verbatim}
    \StartChapter{這是標題}

    \StartSection{節標題}
    內容 ...

    \StartSubSection{小節標題}
    內容 ...

    \StartSubSubSection{小小節標題}
    內容 ...

    \EndChapter
  \end{verbatim}
\end{DescriptionFrame}

\newpage% ------------------------------------------------

\newpage
\StartSection{Figure使用透明度}

\vspace{2.0cm}

\InsertFigure
  [scale=0.5,
    caption={opacity使用預設}]
  {tutorial--pre-PH-fork/abstract/pic/extended-abstract-2.jpg}

\InsertFigure
  [scale=0.5,
    caption={測試opacity=0.4},
    opacity=0.4]
  {tutorial--pre-PH-fork/abstract/pic/extended-abstract-2.jpg}

\newpage

\EmptyLine
\vspace{7.0cm}

    \InsertFigures
    [caption={opacity使用預設}] %
    {
      {tutorial--pre-PH-fork/how-to/write/figure/pic/CC-BY-NC.png}
    }%
    {
      {tutorial--pre-PH-fork/how-to/write/figure/pic/CC-BY-NC-ND.png}
    }

\vspace{1.0cm}

    \InsertFigures
    [caption={測試opacity=0.4},
    opacity=0.4]
    {
      {tutorial--pre-PH-fork/how-to/write/figure/pic/CC-BY-NC.png}
    }%
    {
      {tutorial--pre-PH-fork/how-to/write/figure/pic/CC-BY-NC-ND.png}
    }

% ------------------------------------------------

\newpage% ------------------------------------------------

\newpage
\StartSection{Table使用透明度}

\vspace{3.0cm}

\InsertTable
  [caption={opacity使用預設}]
  {
    \begin{tabular}{llll}
    \hline
    Engine &  &  & OPEL Astra C16SE \\ \hline
    Displacement (cc) &  &  & 1598 \\
    Bore x stroke(mm x mm) &  &  & 79 x 81.5 \\
    Value mechanism &  &  & SOHC \\
    Number of valves &  &  & Intake 4, exhaust 4 \\
    Compression ratio &  &  & 9.8:1 \\
    Torque &  &  & 135/3400 Nm/rpm \\
    Power &  &  & 74/5800 kW/rpm \\
    Ignition sequence &  &  & 1-3-4-2 \\
    Spark plug &  &  & BPR6ES \\
    Fuel &  &  & 95 unleaded gasoline \\
    Cylinder arrangment &  &  & In-line 4 cylinders \\ \hline
    \end{tabular}
  } % End of  \InsertTable{}

  \InsertTable
    [caption={測試opacity=0.4},
      opacity=0.4]
    {
      \begin{tabular}{|c|c|c|c|c|c|c|c|c|c|c|c|c|c|c|c|}
      \hline
       & Col 1 & Col 2 & Col 3 & Col 4 & Col 5 & Col 6 & Col 7 & Col 8 & Col 9 & Col 10 & Col 11 & Col 12 & Col 13 & Col 14 \\ \hline
      Row 1 & Value & Value & Value & Value & Value & Value & Value & Value & Value & Value & Value & Value & Value & Value \\ \hline
      Row 2 & Value & Value & Value & Value & Value & Value & Value & Value & Value & Value & Value & Value & Value & Value \\ \hline
      Row 3 & Value & Value & Value & Value & Value & Value & Value & Value & Value & Value & Value & Value & Value & Value \\ \hline
      Row 4 & Value & Value & Value & Value & Value & Value & Value & Value & Value & Value & Value & Value & Value & Value \\ \hline
      \end{tabular}
  } % End of  \InsertTable{}

% ------------------------------------------------

\newpage
\StartSection{Table測試寬度}

  \InsertTable
    [caption={不進行寬度設定, 應超出頁面}]
    {
      \begin{tabular}{|c|c|c|c|c|c|c|c|c|c|c|c|c|c|c|c|}
      \hline
       & Col 1 & Col 2 & Col 3 & Col 4 & Col 5 & Col 6 & Col 7 & Col 8 & Col 9 & Col 10 & Col 11 & Col 12 & Col 13 & Col 14 \\ \hline
      Row 1 & Value & Value & Value & Value & Value & Value & Value & Value & Value & Value & Value & Value & Value & Value \\ \hline
      Row 2 & Value & Value & Value & Value & Value & Value & Value & Value & Value & Value & Value & Value & Value & Value \\ \hline
      Row 3 & Value & Value & Value & Value & Value & Value & Value & Value & Value & Value & Value & Value & Value & Value \\ \hline
      Row 4 & Value & Value & Value & Value & Value & Value & Value & Value & Value & Value & Value & Value & Value & Value \\ \hline
      \end{tabular}
  } % End of  \InsertTable{}

  \InsertTable
    [scale=0.9,
      caption={表格寬度設定scale=0.9}]
    {
      \begin{tabular}{|c|c|c|c|c|c|c|c|c|c|c|c|c|c|c|c|}
      \hline
       & Col 1 & Col 2 & Col 3 & Col 4 & Col 5 & Col 6 & Col 7 & Col 8 & Col 9 & Col 10 & Col 11 & Col 12 & Col 13 & Col 14 \\ \hline
      Row 1 & Value & Value & Value & Value & Value & Value & Value & Value & Value & Value & Value & Value & Value & Value \\ \hline
      Row 2 & Value & Value & Value & Value & Value & Value & Value & Value & Value & Value & Value & Value & Value & Value \\ \hline
      Row 3 & Value & Value & Value & Value & Value & Value & Value & Value & Value & Value & Value & Value & Value & Value \\ \hline
      Row 4 & Value & Value & Value & Value & Value & Value & Value & Value & Value & Value & Value & Value & Value & Value \\ \hline
      \end{tabular}
  } % End of  \InsertTable{}

% ------------------------------------------------

\newpage% ------------------------------------------------

\newpage
\StartSection{Equation}

\EquationBegin
  x = &a + b + c + \\
  &d + e + f + g + \\
  &h + i + j + k
\EquationEnd

\EquationBegin{testpage:equation:eq1}E = mc^2\EquationEnd

% ------------------------------------------------

\newpage% ------------------------------------------------
\StartSection{術語/符號 Nomenclature}{chapter:how-to:write:nomenclature}
% ------------------------------------------------

Nomenclature在定義一些在整份論文中所會用到的變數是很常用到的. 它的位置會出現在文章當中或是在Chapter 1之前. 它的設計沒有一個標準答案, 在不同的情況下可能有不同顯示方式, 但它基本上跟一張Table是沒差的. 而它在Latex中是使用一個package名為`nomencl'.

但經過研究了一下package `nomencl'或tabbing這些用來建Nomenclature的方式後, 發現`nomencl'在設計上反而會增加在產生論文時的步驟; 而tabbing要自行定義一個闊度才能弄得比較好看, 但同時內容卻出現沒法置中和設計上等一些問題. 故最後決定直接套用Table來讓同學更能自由的設計不同的Nomenclature table.

設計Nomenclature table需要2個知識或工具:\\
1) 設計一張Table, 這邊請參考P. \RefPage{chapter:how-to:write:table}.\\
2) 有關所需要用到的符號, 請參考Equation (P. \RefPage{chapter:how-to:write:equation})中所使用到的工具, Texmarker左邊的工具列, 或看這幾個網頁\RefBib{web:symbols:site1}\RefBib{web:symbols:site2}\RefBib{web:symbols:site3}, 應該已經足夠同學們寫出合適的符號.

% ------------------------------------------------
%\newpage
\StartSubSection{使用方式}

如果是指是在Chapter 1之前的一大張的Nomenclature table, 為Nomenclature Chapter.
  \begin{verbatim}
  \StartNomChapter{ NAME }{ LABEL }
  \EndNomChapter
  \end{verbatim}
Nomenclature Chapter跟一般Chapter的使用方式是一樣的, 但差別在於不會出現`Chapter'這字眼. 而由於大家的Nomenclature Chapter name可能不一樣, 故跟Chapter一樣可設定自行的name.

而如果是在文章當中的Nomenclature table. 基本上就是使用同一個的`\verb|\InsertTable|', 但還可以使用`nomtitle'來設定標題. `nomtitle'跟`caption'的差別是, 使用`nomtitle'所顯示出來的標題是沒有`Table XX:'為開頭, 同樣都是使用`pos'來控制題目的位置.

  \begin{DescriptionFrame}
  \begin{verbatim}
  Options 設定
    nomtitle:   Nomenclature 標題 (選填)
    ...

  E.g
    \InsertTable
    [nomtitle={這是Nomenclature Table的標題}]
      {
        ...
      }
  \end{verbatim}
  \end{DescriptionFrame}

有關這個的用法可參考`tutorial--pre-PH-fork/nomenclature/nomenclature.tex'中的Nomenclature Chapter所demo的例子, 那2個例子只是最簡單的Nomenclature table設計, 應該足夠同學們去弄出合適自己的Nomenclature table的設計.

\newpage% ------------------------------------------------
\StartSection{文獻引用 Bibliography/Reference}{chapter:how-to:write:bib}
% ------------------------------------------------

\StartSubSection{介紹}

Reference對論文來講十分重要的東西, 所以如果你引用的paper數量不少, 那在整理上會有點麻煩, 所以世界上有不少東西來管理這部份的資料, 如用的Word的話會配合Endnote.

而本模版是使用LaTex中的BibTex來管理, 你可以在`./content/references'找到3個`.bib'檔, 那正是你可以把你所引用的內容放在裡面.

Bib的分類滿多 (參考\RefBib{web:latex:bib_manage}), 但論文主要都是引用`book' (課本, 書籍等), `misc' (網頁, 任何其他東西), `inproceedings' (論文類)中的內容, 所以本模版提供的樣板檔案為`book.bib', `misc.bib' 跟 `paper.bib'.

\StartSubSection{使用方式}

任何放置論文的出版社(如ACM, IEEE, DBLP等), 都會為了方便別人去引用, 都會提供一些資料以給放在論文中引用. Fig \RefTo{fig:write:bib:1} 是以ACM Digital Library例子, 簡單說明如何使用BibTex來管理.

\InsertFigure
  [caption={ACM Digital Library例子},
    label={fig:write:bib:1}, scale=0.5]
  {tutorial--pre-PH-fork/how-to/write/bib/pic/1.png}

\InsertFigure
  [caption={BibTex的位置},
    label={fig:write:bib:2}, scale=0.4]
  {tutorial--pre-PH-fork/how-to/write/bib/pic/2.png}

在畫面右方會看到`Export Formats'的位置, 會看到如fig \RefTo{fig:write:bib:2}中一個的BibTex的按鈕.

\InsertFigure
  [caption={BibTex資料},
    label={fig:write:bib:3}, scale=0.5]
  {tutorial--pre-PH-fork/how-to/write/bib/pic/3.png}

按它後就會出現如fig \RefTo{fig:write:bib:3}這個畫面, 這個就是要填進Bib的資料, 所以把這個東西複製到Bib檔內.

\InsertFigure
  [caption={整理/使用BibTex},
    label={fig:write:bib:4}, scale=0.5]
  {tutorial--pre-PH-fork/how-to/write/bib/pic/4.png}

但複製完後要改一個東西, 第一行是所謂的label部份(參考Chap \RefTo{chapter:how-to:write:label}), 所以要改成一個自己能記得的label以方便在內容中來引用.

%有什麼問題可以去問Google\cite{website:google}老師. (如果有設定references用的檔案, 即使用了ReferencesFiles, 那必須至少要存在一個cite才不會顯示錯誤.)

% ------------------------------------------------
\EndChapter
% ------------------------------------------------

\newpage% ------------------------------------------------
\StartSection{虛擬程式碼(Pseudocode)}{chapter:how-to:write:pseudocode}
% ------------------------------------------------

Pseudocode在資訊類的paper是很常見, 雖然這東西冷門, 但是有它的存在意義.
如果不想用Pseudocode來寫, 可考慮使用Table來做 (考慮章節 \RefTo{subsection:how-to:write:table:api}).

而由於需要寫Pseudocode的人, 理論上都100\%會寫程式, 所以有關這邊會直接使用例子(基本的function, if-elseif-else, while, return, switch-case)來說明, 靠例子應該就能寫出你所要的Pseudocode.

唯一注意的是需要使用:\\
'\verb|\Statex|'來斷一行空行\\
'\verb|\State|'來斷一行以寫新code在後面

% ------------------------------------------------

\newpage
例子1:
\begin{algorithm}
  \caption{My algorithm (function A)}
  \label{algo:functionA}

  \begin{algorithmic}[1]
    \Function{function\_name\_a}{arg1, arg2}
      \If{conditionA}
        \State ...
      \ElsIf{conditionB}
        \State ...
      \Else
        \State ...
      \EndIf
      \Statex
      \If{condition1}
        \State ...
      \Else
        \If{condition2}
          \State ...
        \Else
          \State ...
        \EndIf
      \EndIf
      \Statex
      \For{condition}
        \State ...
      \EndFor
    \EndFunction
  \end{algorithmic}
\end{algorithm}

\newpage
針對function A (Algorithm \RefTo{algo:functionA}), 它的LaTex寫法為:\\

\begin{DescriptionFrame}
  \begin{verbatim}
\begin{algorithm}
  \caption{My algorithm (function A)}
  \label{algo:functionA}

  \begin{algorithmic}[1]
    \Function{function\_name\_a}{arg1, arg2}
      \If{conditionA}
        \State ...
      \ElsIf{conditionB}
        \State ...
      \Else
        \State ...
      \EndIf
      \Statex
      \If{condition1}
        \State ...
      \Else
        \If{condition2}
          \State ...
        \Else
          \State ...
        \EndIf
      \EndIf
      \Statex
      \For{condition}
        \State ...
      \EndFor
    \EndFunction
  \end{algorithmic}
\end{algorithm}
  \end{verbatim}
\end{DescriptionFrame}

% ------------------------------------------------

\newpage
例子2:
\begin{algorithm}
  \caption{My algorithm (function B)}
  \label{algo:functionB}

  \begin{algorithmic}[1]
    \Function{functionNameB}{}
      \State ...
      \State Some code here
      \State ...
      \Statex
      \While{condition3}
        \State ...
      \EndWhile
      \Statex
      \Repeat
        \State ...
      \Until{condition3}
      \Statex
      \Switch{condition4}
        \Case{condition5} ... \Break \EndCase
        \Statex
        \Case{condition6}
          \State ...
          \State \Break
        \EndCase
        \Statex
        \Default
          \State ...
        \EndDefault
      \EndSwitch

      \Statex\State \Return retValue
    \EndFunction
  \end{algorithmic}
\end{algorithm}

\newpage
針對function B (Algorithm \RefTo{algo:functionB}), 它的LaTex寫法為:\\

\begin{DescriptionFrame}
  \begin{verbatim}
\begin{algorithm}
  \caption{My algorithm (function B)}
  \label{algo:functionB}

  \begin{algorithmic}[1]
    \Function{functionNameB}{}
      \State ...
      \State Some code here
      \State ...
      \Statex
      \While{condition3}
        \State ...
      \EndWhile
      \Statex
      \Repeat
        \State ...
      \Until{condition3}
      \Statex
      \Switch{condition4}
        \Case{condition5} ... \Break \EndCase
        \Statex
        \Case{condition6}
          \State ...
          \State \Break
        \EndCase
        \Statex
        \Default
          \State ...
        \EndDefault
      \EndSwitch

      \Statex\State \Return retValue
    \EndFunction
  \end{algorithmic}
\end{algorithm}
  \end{verbatim}
\end{DescriptionFrame}


% ------------------------------------------------
\EndChapter
% ------------------------------------------------
