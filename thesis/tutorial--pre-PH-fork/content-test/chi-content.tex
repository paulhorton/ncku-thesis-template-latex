% ------------------------------------------------

\newpage
\StartSection{中文內容 (只用段落來分段)}
用來看內容, 符號, 段距, 字元之間的距離等東西\\

本校創校於西元1931年(昭和6年,民國20年)1月15日,原名為「臺南高等工業學校」;1944年(昭和19年,民國33年)改稱為「臺南工業專門學校」。民國34年臺灣光復。本校於民國35年2月改制為「臺灣省立臺南工業專科學校」,由王石安博士擔任校長;35年10月改制為「臺灣省立工學院」,仍由王石安博士擔任校長。彼時僅有成功校區,39年增購勝利校區。41年2月,由秦大鈞博士接任校長。

45年8月,本校改制為「臺灣省立成功大學」,仍由秦大鈞博士擔任校長;同時增設文理學院及商學院。46年8月,由閻振興博士接任校長。54年1月,由羅雲平博士接任校長。55年增購光復校區。58年10月,將文理學院分為文學院及理學院。60年8月,改制為「國立成功大學」,並由倪超博士接任校長;同年增購建國校區。67年8月,由王唯農博士接任校長。

69年8月,夏漢民博士接任校長;同年將商學院更名為管理學院。72年8月,增設醫學院,並增購自強校區及敬業校區。74年增購力行校區部分校地。76年增購歸仁校區,設置航空太空實驗場。77年6月本校醫學院附設醫院正式營運。77年8月,馬哲儒博士接任校長。80年增購自強校區北半部。82年陸續增購台南市「文大五」用地,闢為本校安南校區。

83年8月,吳京院士接任校長。吳校長於85年6月入閣擔任教育部長,由副校長黃定加博士代理校長。86年2月,翁政義博士接任校長。86年8月增設社會科學院。88年完成收購安南校區全部校地;同年6月增購原陸軍八○四醫院用地(91年6月完成撥用手續)。翁校長於89年5月出任國科會主委,由副校長翁鴻山博士代理校長。90年2月,高強博士接任校長。92年8月,增設電機資訊學院、規劃與設計學院。94年7月,本校配合國軍斗六醫院精實案,接管該院營運權,設置為雲林縣斗六校區,並改制為本校醫學院附設醫院斗六分院。94年8月,增設生物科學與科技學院。94年10月,本校得到教育部的肯定,獲選為「發展國際一流大學及頂尖研究中心計畫」的兩所重點大學之一。

96年2月,賴明詔院士接任校長,97年2月教育部公佈「發展國際一流大學及頂尖研究中心計畫」第二梯次的審議結果,本校繼續獲得教育部的肯定與補助,積極朝國際一流大學的目標邁進。97年10月增購歸仁校區北側台糖土地(正式登記為本校管有)。100年2月,黃煌煇博士接任校長。100年4月本校獲得教育部第二期頂尖大學計畫補助,持續朝國際一流大學的目標邁進。104年2月,蘇慧貞博士接任校長。

% ------------------------------------------------

\newpage
\StartSection{中文內容 (只用強制斷行)}
用來看內容, 符號, 段距, 字元之間的距離等東西\\

本校創校於西元1931年(昭和6年,民國20年)1月15日,原名為「臺南高等工業學校」;1944年(昭和19年,民國33年)改稱為「臺南工業專門學校」。民國34年臺灣光復。本校於民國35年2月改制為「臺灣省立臺南工業專科學校」,由王石安博士擔任校長;35年10月改制為「臺灣省立工學院」,仍由王石安博士擔任校長。彼時僅有成功校區,39年增購勝利校區。41年2月,由秦大鈞博士接任校長。\\
45年8月,本校改制為「臺灣省立成功大學」,仍由秦大鈞博士擔任校長;同時增設文理學院及商學院。46年8月,由閻振興博士接任校長。54年1月,由羅雲平博士接任校長。55年增購光復校區。58年10月,將文理學院分為文學院及理學院。60年8月,改制為「國立成功大學」,並由倪超博士接任校長;同年增購建國校區。67年8月,由王唯農博士接任校長。\\
69年8月,夏漢民博士接任校長;同年將商學院更名為管理學院。72年8月,增設醫學院,並增購自強校區及敬業校區。74年增購力行校區部分校地。76年增購歸仁校區,設置航空太空實驗場。77年6月本校醫學院附設醫院正式營運。77年8月,馬哲儒博士接任校長。80年增購自強校區北半部。82年陸續增購台南市「文大五」用地,闢為本校安南校區。\\
83年8月,吳京院士接任校長。吳校長於85年6月入閣擔任教育部長,由副校長黃定加博士代理校長。86年2月,翁政義博士接任校長。86年8月增設社會科學院。88年完成收購安南校區全部校地;同年6月增購原陸軍八○四醫院用地(91年6月完成撥用手續)。翁校長於89年5月出任國科會主委,由副校長翁鴻山博士代理校長。90年2月,高強博士接任校長。92年8月,增設電機資訊學院、規劃與設計學院。94年7月,本校配合國軍斗六醫院精實案,接管該院營運權,設置為雲林縣斗六校區,並改制為本校醫學院附設醫院斗六分院。94年8月,增設生物科學與科技學院。94年10月,本校得到教育部的肯定,獲選為「發展國際一流大學及頂尖研究中心計畫」的兩所重點大學之一。\\
96年2月,賴明詔院士接任校長,97年2月教育部公佈「發展國際一流大學及頂尖研究中心計畫」第二梯次的審議結果,本校繼續獲得教育部的肯定與補助,積極朝國際一流大學的目標邁進。97年10月增購歸仁校區北側台糖土地(正式登記為本校管有)。100年2月,黃煌煇博士接任校長。100年4月本校獲得教育部第二期頂尖大學計畫補助,持續朝國際一流大學的目標邁進。104年2月,蘇慧貞博士接任校長。

% ------------------------------------------------

\newpage
\StartSection{中文內容 (段落 + 強制斷行)}
用來看內容, 符號, 段距, 字元之間的距離等東西\\

本校創校於西元1931年(昭和6年,民國20年)1月15日,原名為「臺南高等工業學校」;1944年(昭和19年,民國33年)改稱為「臺南工業專門學校」。民國34年臺灣光復。本校於民國35年2月改制為「臺灣省立臺南工業專科學校」,由王石安博士擔任校長;35年10月改制為「臺灣省立工學院」,仍由王石安博士擔任校長。彼時僅有成功校區,39年增購勝利校區。41年2月,由秦大鈞博士接任校長。\\

45年8月,本校改制為「臺灣省立成功大學」,仍由秦大鈞博士擔任校長;同時增設文理學院及商學院。46年8月,由閻振興博士接任校長。54年1月,由羅雲平博士接任校長。55年增購光復校區。58年10月,將文理學院分為文學院及理學院。60年8月,改制為「國立成功大學」,並由倪超博士接任校長;同年增購建國校區。67年8月,由王唯農博士接任校長。\\

69年8月,夏漢民博士接任校長;同年將商學院更名為管理學院。72年8月,增設醫學院,並增購自強校區及敬業校區。74年增購力行校區部分校地。76年增購歸仁校區,設置航空太空實驗場。77年6月本校醫學院附設醫院正式營運。77年8月,馬哲儒博士接任校長。80年增購自強校區北半部。82年陸續增購台南市「文大五」用地,闢為本校安南校區。\\

83年8月,吳京院士接任校長。吳校長於85年6月入閣擔任教育部長,由副校長黃定加博士代理校長。86年2月,翁政義博士接任校長。86年8月增設社會科學院。88年完成收購安南校區全部校地;同年6月增購原陸軍八○四醫院用地(91年6月完成撥用手續)。翁校長於89年5月出任國科會主委,由副校長翁鴻山博士代理校長。90年2月,高強博士接任校長。92年8月,增設電機資訊學院、規劃與設計學院。94年7月,本校配合國軍斗六醫院精實案,接管該院營運權,設置為雲林縣斗六校區,並改制為本校醫學院附設醫院斗六分院。94年8月,增設生物科學與科技學院。94年10月,本校得到教育部的肯定,獲選為「發展國際一流大學及頂尖研究中心計畫」的兩所重點大學之一。\\

96年2月,賴明詔院士接任校長,97年2月教育部公佈「發展國際一流大學及頂尖研究中心計畫」第二梯次的審議結果,本校繼續獲得教育部的肯定與補助,積極朝國際一流大學的目標邁進。97年10月增購歸仁校區北側台糖土地(正式登記為本校管有)。100年2月,黃煌煇博士接任校長。100年4月本校獲得教育部第二期頂尖大學計畫補助,持續朝國際一流大學的目標邁進。104年2月,蘇慧貞博士接任校長。

% ------------------------------------------------
